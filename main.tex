\documentclass[12pt,english,brazil,a4paper,utf8,oneside]{utfpr-tcc}

% Este comando não é necessário: utilizei apenas para deixar o latex2rtf
% feliz (e descobrir a codificação do texto).
\usepackage[utf8]{inputenc}

% Suporte a figuras e subfiguras
\usepackage{graphics}
\usepackage{subfigure}

% Suporte a tabelas (principalmente do cronograma)
\usepackage{tabularx}
\usepackage{multirow}
\usepackage{array}
\usepackage{colortbl}
\usepackage{hhline}
\usepackage{xcolor}
\usepackage{indentfirst}
\usepackage{lscape}
\usepackage{appendix}

% Elementos geralmente utilizados na tabela do cronograma
\newcommand{\fullcell}{\multicolumn{1}{>{\columncolor[gray]{0.5}}c}{}}
\newcommand{\fullcellline}{\multicolumn{1}{>{\columncolor[gray]{0.5}}c|}{}}
\newcommand{\mc}[3]{\multicolumn{#1}{#2}{#3}}
\newcommand{\y}{\rule{8pt}{4pt}}
\newcommand{\n}{\hspace*{8pt}} 

% Define o caminho das figuras
\graphicspath{{images/}}

% Dados do curso que não precisam de alteração
\university{Universidade Tecnológica Federal do Paraná}
\universityen{Federal University of Technology -- Paraná}
\universityunit{Departamento Acadêmico de Computação}
\address{Campo Mourão}
\addressen{Campo Mourão, PR, Brazil}
\documenttype{Monografia}
\documenttypeen{Monograph}
\degreetype{Graduação}


%%%%%%%%%%%%%%%%%%%%%%%%%%%%%%%%%%%%%%%%%%%%%%%%%%%%%%%%%%%%%%%%%%%%%%%%%%%%%
% Alterar daqui para baixo
%%%%%%%%%%%%%%%%%%%%%%%%%%%%%%%%%%%%%%%%%%%%%%%%%%%%%%%%%%%%%%%%%%%%%%%%%%%%%

% Dados do curso. Caso seja BCC:
\program{Curso de Bacharelado em Ciência da Computação}
\programen{Undergradute Program in Computer Science}
\degree{Bacharel}
\degreearea{Ciência da Computação}

% Dados da disciplina. Escolha uma das opções e a descomente:
% TCC1:
\goal{Proposta de Trabalho de Conclusão de Curso de Graduação}
\course{Trabalho de Conclusão de Curso 1}
% TCC2:
% \goal{Trabalho de Conclusão de Curso de graduação}
% \course{Trabalho de Conclusão de Curso 2}


% Dados do TCC (precisa alterar)
\author{Emerson Yudi Nakashima}  % Seu nome
\title{Avaliação de Programas em MOOC com Emprego de Técnicas de Visualização} % Título do trabalho
\titleen{} % Título traduzido para inglês
\advisor{Prof. Dr. Marco Aurélio Graciotto Silva} % Nome do orientador. Lembre-se de prefixar com "Prof. Dr.", "Profª. Drª.", "Prof. Me." ou "Profª. Me."}
\coadvisor{Profª. Drª. Aretha Barbosa Alencar} % Nome do coorientador, caso exista. Caso não exista, comente a linha.
\depositshortdate{2016} % Ano em que depositou este documento

% Dados da ficha catalografica. Ela é opcional, mas é uma boa ideia inserí-la. Exemplos para geração (http://fichacatalografica.sibi.ufrj.br/)
\fichacatautor{Nakashima, Emerson Y}  % Nome conforme citado (ou seja, no formato "Sobrenome, Nome").
\fichacatbib{Biblioteca da UTFPR de Campo Mourão} % Não alterar
\fichacatpum{N163} % Código Cutter-Sanborn. Use a primeira letra do sobrenome seguido do número conforme as primeiras letras do sobrenome e a tabela http://www.amormino.com.br/cutter-sanborn/cutter1.html
\fichacatpalcha{} % Assuntos do trabalho. Cada item deve ser enumerado e separado por ponto: 1. xxx. 2. yyy. 3. zzz.
\fichacatpdois{} % Deixar em branco


\begin{document}
	
\frontmatter
\maketitle

%\begin{resumo}
%% TODO: se possível, escreva um resumo estruturado. Para TCC 1, o resumo estruturado teria os seguintes elementos:
%\textbf{Contexto:} Em disciplinas de algoritmos ou programação, é necessário criar
%e implementar uma solução para um problema. Entretanto, há a possibilidade de possuir
%diversas formas de solucionar o problema e, consequentemente, formas de implementações.
%Desta forma, a quantidade de implementações possíveis é vasta, dificultando a avaliação
%delas pelo professor quanto ao custo, tempo e qualidade da avaliação. Além disso, é
%necessário realizar um \foreign{feedback} para o aluno com relação a sua implementação.
%Para agravar essa dificuldade, cursos massivos, abertos e \foreign{online} (MOOC)
%possuem uma grande quantidade de usuários, agravando este problema e inviabilizando
%a correção individual das submissões.
%
%\textbf{Objetivo:} O objetivo deste trabalho é propor subsídios para a avaliação de
%programas submetidos em disciplinas introdutórias à computação, utilizando técnicas
%de mineração e visualização de dados para construir e apresentar agrupamentos de
%código-fonte semelhantes. Os subsídios propostos consistem na utilização de ferramentas
%para extração de características, padronização no armazenamento dessas características
%e a utilização de técnicas agrupamento e visualização, com o auxílio de uma ferramenta.
%
%\textbf{Método:} A primeira etapa consistirá na identificação das características que
%podem ser extraídas conforme o tipo de análise utilizado: características estática, do
%estilo de escrita e dinâmica. Após a identificação, será necessário o desenvolvimento
%de ferramentas para coletar tais medidas de forma que possamos utilizá-las para realizar
%o agrupamento. Com isso, desenvolveremos uma ferramenta para minerar as informações
%disponíveis, realizando os agrupamentos e gerando uma visualização dos programas submetidos.
%Para testar a ferramenta, teremos duas bases de dados distintas de implementações. Uma
%base de dados existente com soluções de cinco problemas distintos e outra base de dados
%com códigos-fontes de alunos. A validação ocorrerá em duas etapas: mediante a qualidade
%das visualizações considerando as técnicas de mineração e visualização de dados; e
%verificando se o \foreign{feedback} está sendo construtivo ao aluno conforme seu
%\foreign{feedback} e da qualidade das próximas implementações.
%
%\textbf{Resultados esperados:} Espera-se que, com a utilização dos subsídios logrados
%neste trabalho, potencializar a correção das submissões de modo que, principalmente,
%melhore a qualidade do \foreign{feedback} podendo levar menos tempo para corrigi-las
%ou o mesmo tempo que a correção tradicional.
%
%% ou, para TCC 2:
%% \textbf{Contexto:} \\
%% \textbf{Objetivo:} \\
%% \textbf{Método:} \\
%% \textbf{Resultados:} \\
%% \textbf{Conclusões:}
%
%% Palavras-chaves, separadas por ponto (tente não definir mais do que cinco)
%\palavraschaves{MOOC. Programação. Agrupamentos. Mineração. Visualização}
%\end{resumo}



% Caso seja TCC 2, precisa traduzir o resumo e as palavras-chaves para inglês:
% \begin{abstract}
% \textbf{Context:}
% \textbf{Objective:}
% \textbf{Method:}
% \textbf{Results:}
% \textbf{Conclusions:}

% Palavras-chaves em inglês, separadas por ponto.
% \keywords{}
% \end{abstract}



% Listas (opcionais, mas recomenda-se a partir de 5 elementos)
\listoffigures
\listoftables
\listofacronyms{acronimos}

% Sumário
\tableofcontents

\mainmatter
% TODO: incluir arquivos latex com os capítulos
\chapter{Introdução}

	\ac{MOOC} é uma plataforma de ensino online
	com cursos massivos e abertos que visa oferecer os mais diversos cursos a nível
	global. Para realizar um curso ofertado pelo \acs{MOOC}, é necessário apenas a conexão
	com a Internet e o cadastro na plataforma, geralmente gratuito. Foi popularizado
	em 2011 quando grandes universidades, como o Instituto de Tecnologia de
	Massachusetts, a Universidade de Harvard e a Universidade de Stanford tiveram
	a iniciativa mediados pelos provedores Cousera, edX e Udacity, respectivamente
	\cite{Mehlenbacher:2012}. Além de permitir explorar novos modelos de negócio
	\cite{dellarocas2013money}, possibilita englobar mais alunos a custos menores
	e com boa qualidade \cite{schmidt2013producing}.
	
	Em cursos de introdução a programação, seja \acs{MOOC} ou presencial, há uma introdução
	sobre algoritmo seguido da seleção de uma linguagem de programação para
	desenvolvimento. Alguns desses cursos orientam a utilização de ferramentas, como
	o \ac{IDE} e ensinam o funcionamento da linguagem sintaticamente.
	Entretanto, esse tipo de ensino tem levado o estudante a fazer seu programa baseado
	na tentativa e erro, visto que buscam gerar o algoritmo sem o total conhecimento
	lógico do problema \cite{edwards2003}. Isso funcionou como um incentivo para que
	surgissem outras abordagens de ensino, como a reflexão na ação (\foreign{Reflection
	in action}) e o \ac{TDD} \cite{camara_graciottoSilva2016}. Na primeira abordagem, a
	implementação do \foreign{software} ocorre somente após o entendimento apropriado
	do problema \cite{edwards2004}. No \acs{TDD}, cumpre-se um ciclo dividido em três etapas:
	na primeira etapa o programador insere um teste que deve falhar no momento de sua
	execução; na segunda etapa é implementado a solução para que o caso de teste
	escrito anteriormente seja aceito; e a última etapa refere-se a refatoração
	(simplificar ou melhorar) o código recém implementado \cite{beck2003}.
	
	Nesses cursos são necessários criar e implementar uma solução para um problema.
	Entretanto, há a possibilidade de possuir diversas formas de solucionar o problema
	e, consequentemente, formas de implementações. Desta forma, a quantidade de
	implementações possíveis é vasta, dificultando a avaliação delas pelo professor
	quanto ao custo, tempo e qualidade da avaliação. Além disso, é necessário realizar
	um \foreign{feedback} para o aluno com relação a sua implementação. Para agravar
	essa dificuldade, cursos massivos, abertos e \foreign{online} (MOOC) possuem uma
	grande quantidade de usuários, agravando este problema e inviabilizando a correção
	individual das submissões.

	Em cursos de introdução a programação, seja \acs{MOOC} ou presencial, há uma introdução
	sobre algoritmo seguido da seleção de uma linguagem de programação para
	desenvolvimento. Alguns desses cursos orientam a utilização de ferramentas, como
	o \ac{IDE} e ensinam o funcionamento da linguagem sintaticamente.
	Entretanto, esse tipo de ensino tem levado o estudante a fazer seu programa baseado
	na tentativa e erro, visto que buscam gerar o algoritmo sem o total conhecimento
	lógico do problema \cite{edwards2003}. Isso funcionou como um incentivo para que
	surgissem outras abordagens de ensino, como a reflexão na ação (\foreign{Reflection
	in action}) e o \ac{TDD} \cite{camara_graciottoSilva2016}. Na primeira abordagem, a
	implementação do \foreign{software} ocorre somente após o entendimento apropriado
	do problema \cite{edwards2004}. No \acs{TDD}, cumpre-se um ciclo dividido em três etapas:
	na primeira etapa o programador insere um teste que deve falhar no momento de sua
	execução; na segunda etapa é implementado a solução para que o caso de teste
	escrito anteriormente seja aceito; e a última etapa refere-se a refatoração
	(simplificar ou melhorar) o código recém implementado \cite{beck2003}.

\chapter{Delimitação}

	Esse estudo consiste na utilização de fundamentos das áreas de: Mineração de
	dados, Inteligência Artificial e Visualização de Dados. A Mineração de dados
	é utilizada por meio do reconhecimento e extração de características do código-fonte.
	Após a padronização de um formato de apresentação dessas características, é
	necessário o uso de algoritmos não supervisionados de Inteligência Artificial para
	que seja realizado o agrupamento das implementações possívelmente semelhantes.
	Com isso, emprega-se técnicas de Visualização de Dados para que seja possível
	representar os agrupamentos visualmente, por meio de projeções multidimensionais
	e algoritmos de visualização.
\chapter{Problemas e premissas}
\label{chapter:problemas-premissas}

Principalmente em \acs{MOOC} de ensino de programação, tratando-se de uma ferramenta de
ensino a nível mundial, deve-se considerar um grande número de usuários. Com isso,
torna-se necessário a utilização de mecanismos de avaliação automática ou
semiautomática \cite{schmidt2013producing}. Para alguns tipos de atividade, a
avaliação automática é bem simples. Por exemplo, para verificar as soluções
referentes às questões de múltipla escolha é necessário somente comparar a
alternativa selecionada com o gabarito de questões \cite{alario2013analysing}. No
entanto, implementações de programas computacionais necessitam que seus algoritmos
sejam analisados quanto às saídas geradas pela sua execução, projeto do algoritmo,
facilidade de compreensão, dentro outros requisitos, a fim de retornar o resultado
da avaliação. Como há uma grande quantidade de submissões de trabalhos em cursos
de programação, consequentemente acarreta alguns problemas como: avaliar todas as
submissões, tempo e aumento de custo de correção por parte do professor.

Considerando que muito projetos podem ser semelhantes, os professores podem explorar
e compreender as variações de implementações \cite{Yin:2015}, a fim de diminuir
o tempo gasto para correção dos códigos-fontes submetidos, por meio de agrupamentos
(\foreign{clusters}). Por exemplo, tais agrupamentos podem ser realizados por meio
da similaridade dos códigos. Compreendendo as variações de implementação, é possível
extrair caraterísticas por meio da análise estática \cite{Yin:2015,Glassman:2014,Taherkhani:2012},
análise dinâmica \cite{Glassman:2015} e análise do estilo de escrita \cite{Wei2015}.
Com	isso, o agrupamento das submissões semelhantes infere na correção efetiva de
poucos projetos, já que todos os outros códigos que estão no agrupamento serão
parecidos, permitindo que o professor dedique o tempo poupado na correção de diversas
implementações para aprimorar as correções de poucos trabalhos, possibilitando
\foreign{feedbacks} mais precisos e menor custo para o \acs{MOOC}.

\chapter{Objetivos}

Considerando que muito projetos podem ser semelhantes, os professores podem explorar
e compreender as variações de implementações \cite{Yin:2015}, a fim de diminuir
o tempo gasto para correção dos códigos-fontes submetidos, por meio de agrupamentos
(\foreign{clusters}). Por exemplo, tais agrupamentos podem ser realizados por meio
da similaridade dos códigos. Compreendendo as variações de implementação, é possível
extrair caraterísticas por meio da análise estática \cite{Yin:2015,Glassman:2014,Taherkhani:2012},
análise dinâmica \cite{Glassman:2015} e análise do estilo de escrita \cite{Wei2015}.
Com	isso, o agrupamento das submissões semelhantes infere na correção efetiva de
poucos projetos, já que todos os outros códigos que estão no agrupamento serão
parecidos, permitindo que o professor dedique o tempo poupado na correção de diversas
implementações para aprimorar as correções de poucos trabalhos, possibilitando
\foreign{feedbacks} mais precisos e menor custo para o \acs{MOOC}.

Considerando este cenário, a concretização dos objetivos deste trabalho gerará
os seguintes subsídios para a avaliação de programas submetidos em \acs{MOOC}:  % TODO: a concretização das metas gerarão/são os resultados/subsídios definidos no título do trabalho. Podemos deixar as metas mais claras nesse sentido e colocá-las como itemize.
\begin{itemize}
	\item Agrupamento dos códigos-fontes semelhantes, utilizando técnicas de mineração;
	\item Técnica de projeção para mapeamento dos \foreign{clusters};
	\item Ferramenta de visualização;
	\item Material para os professores sobre como corrigir as submissões utilizando a ferramenta.
\end{itemize}

Por meio desses subsídios buscaremos diminuir o tempo gasto na correção de todos os códigos
fontes e  fornecer \foreign{feedbacks} construtivos de modo que o usuário consiga
corrigir seus erros e submeta novamente seu algoritmo. Como metas, estabeleceu-se
o desenvolvimento de uma ferramenta para recuperação de dados a partir de
códigos-fontes e da alteração de uma ferramenta para mineração e visualização de
dados \cite{Alencar-etal:2012}. Com o auxílio dessas ferramentas, serão investigadas as
características e técnicas para mineração e visualização dos programas submetidos,
avaliando-se como contribuir para a correção dos trabalhos submetidos, o tempo em
que foi necessário para que os professores corrigissem todas as submissões, a
qualidade dos agrupamentos realizados pelo sistema e, principalmente, a
qualidade do \foreign{feedback}.
\chapter{Justificativa}

A conclusão desse estudo pode impactar três envolvidos distintos: o professor;
a instituição que administra o \acs{MOOC}; e o usuário ou aluno. O professor
corrigirá menos submissões, devido ao agrupamento das implementações semelhantes,
além de poder corrigir com maior minuciosidade a fim de melhorar as correções e
realizar melhores \foreign{feedbacks} aos usuários. Pelo fato de corrigir uma
quantidade consideravelmente menor de códigos-fontes, o professor poderá levar
menos tempo para corrigí-las. Com isso, a instituição pagará uma quantidade menor
de dias trabalhados. E o aluno receberá \foreign{feedbacks} bem elaborados,
impactando diretamente na sua aprendizagem.
\chapter{Proposta}
\label{chap:Proposta}
\chapter{Cronograma}

	A \cref{tab:cronograma} apresenta o cronograma para conclusão do Trabalho de
	Conclusão de Curso.
	
	\begin{table}[]
		\begin{tabular}{|l|c|c|c|c|c|c|}
			\hline
			& Janeiro & Fevereiro & Março & Abril & Maio & Junho \\ \hline
			Implementação da \texttt{ScienceView} 		& \y  & \y  & \y  &     &     &     \\ \hline
			Construção da base de dados      		        &     & \y  & \y  &     &     &     \\ \hline
			Extração de caraterísticas         		        &     &     & \y  & \y  &     &     \\ \hline
			Validação dos agrupamentos      		        &     &     &     & \y  & \y  & \y  \\ \hline
			Validação dos \foreign{feedbacks}               &     &     &     & \y  & \y  & \y  \\ \hline
			Publicação dos resultados                       &     &     & \y  & \y  & \y  & \y  \\ \hline
		\end{tabular}
		\caption{Cronograma}
		\label{tab:cronograma}
	\end{table}

\bibliographystyle{abntex2-alf}
\bibliography{main} % geração automática das referências a partir do arquivo main.bib

\backmatter
% \appendix


\end{document}
