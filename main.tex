\documentclass[12pt,english,brazil,a4paper,utf8,oneside]{utfpr-tcc}

% Este comando não é necessário: utilizei apenas para deixar o latex2rtf
% feliz (e descobrir a codificação do texto).
\usepackage[utf8]{inputenc}

% Suporte a figuras e subfiguras
\usepackage{graphics}
\usepackage{subfigure}

% Suporte a tabelas (principalmente do cronograma)
\usepackage{tabularx}
\usepackage{multirow}
\usepackage{array}
\usepackage{colortbl}
\usepackage{hhline}
\usepackage{xcolor}
\usepackage{indentfirst}
\usepackage{lscape}
\usepackage{appendix}

% Elementos geralmente utilizados na tabela do cronograma
\newcommand{\fullcell}{\multicolumn{1}{>{\columncolor[gray]{0.5}}c}{}}
\newcommand{\fullcellline}{\multicolumn{1}{>{\columncolor[gray]{0.5}}c|}{}}
\newcommand{\mc}[3]{\multicolumn{#1}{#2}{#3}}
\newcommand{\y}{\rule{8pt}{4pt}}
\newcommand{\n}{\hspace*{8pt}} 

% Define o caminho das figuras
\graphicspath{{images/}}

% Dados do curso que não precisam de alteração
\university{Universidade Tecnológica Federal do Paraná}
\universityen{Federal University of Technology -- Paraná}
\universityunit{Departamento Acadêmico de Computação}
\address{Campo Mourão}
\addressen{Campo Mourão, PR, Brazil}
\documenttype{Monografia}
\documenttypeen{Monograph}
\degreetype{Graduação}


%%%%%%%%%%%%%%%%%%%%%%%%%%%%%%%%%%%%%%%%%%%%%%%%%%%%%%%%%%%%%%%%%%%%%%%%%%%%%
% Alterar daqui para baixo
%%%%%%%%%%%%%%%%%%%%%%%%%%%%%%%%%%%%%%%%%%%%%%%%%%%%%%%%%%%%%%%%%%%%%%%%%%%%%

% Dados do curso. Caso seja BCC:
\program{Curso de Bacharelado em Ciência da Computação}
\programen{Undergradute Program in Computer Science}
\degree{Bacharel}
\degreearea{Ciência da Computação}

% Dados da disciplina. Escolha uma das opções e a descomente:
% TCC1:
\goal{Proposta de Trabalho de Conclusão de Curso de Graduação}
\course{Trabalho de Conclusão de Curso 1}
% TCC2:
% \goal{Trabalho de Conclusão de Curso de graduação}
% \course{Trabalho de Conclusão de Curso 2}


% Dados do TCC (precisa alterar)
\author{Emerson Yudi Nakashima}  % Seu nome
\title{Avaliação de Programas em MOOC com Emprego de Técnicas de Visualização} % Título do trabalho
\titleen{} % Título traduzido para inglês
\advisor{Prof. Dr. Marco Aurélio Graciotto Silva} % Nome do orientador. Lembre-se de prefixar com "Prof. Dr.", "Profª. Drª.", "Prof. Me." ou "Profª. Me."}
\coadvisor{Profª. Drª. Aretha Barbosa Alencar} % Nome do coorientador, caso exista. Caso não exista, comente a linha.
\depositshortdate{2016} % Ano em que depositou este documento

% Dados da ficha catalografica. Ela é opcional, mas é uma boa ideia inserí-la. Exemplos para geração (http://fichacatalografica.sibi.ufrj.br/)
\fichacatautor{Nakashima, Emerson Y}  % Nome conforme citado (ou seja, no formato "Sobrenome, Nome").
\fichacatbib{Biblioteca da UTFPR de Campo Mourão} % Não alterar
\fichacatpum{N163} % Código Cutter-Sanborn. Use a primeira letra do sobrenome seguido do número conforme as primeiras letras do sobrenome e a tabela http://www.amormino.com.br/cutter-sanborn/cutter1.html
\fichacatpalcha{} % Assuntos do trabalho. Cada item deve ser enumerado e separado por ponto: 1. xxx. 2. yyy. 3. zzz.
\fichacatpdois{} % Deixar em branco


\begin{document}
	
\frontmatter
\maketitle

\begin{resumo}
% TODO: se possível, escreva um resumo estruturado. Para TCC 1, o resumo estruturado teria os seguintes elementos:
\textbf{Contexto:} Em disciplinas de algoritmos ou programação, é necessário criar
e implementar uma solução para um problema. Entretanto, há a possibilidade de possuir
diversas formas de solucionar o problema e, consequentemente, formas de implementações.
Desta forma, a quantidade de implementações possíveis é vasta, dificultando a avaliação
delas pelo professor quanto ao custo, tempo e qualidade da avaliação. Além disso, é
necessário realizar um \foreign{feedback} para o aluno com relação a sua implementação.
Para agravar essa dificuldade, cursos massivos, abertos e \foreign{online} (MOOC)
possuem uma grande quantidade de usuários, agravando este problema e inviabilizando
a correção individual das submissões.

\textbf{Objetivo:} O objetivo deste trabalho é propor subsídios para a avaliação de
programas submetidos em disciplinas introdutórias à computação, utilizando técnicas
de mineração e visualização de dados para construir e apresentar agrupamentos de
código-fonte semelhantes. Os subsídios propostos consistem na utilização de ferramentas
para extração de características, padronização no armazenamento dessas características
e a utilização de técnicas agrupamento e visualização, com o auxílio de uma ferramenta.

\textbf{Método:} A primeira etapa consistirá na identificação das características que
podem ser extraídas conforme o tipo de análise utilizado: características estática, do
estilo de escrita e dinâmica. Após a identificação, será necessário o desenvolvimento
de ferramentas para coletar tais medidas de forma que possamos utilizá-las para realizar
o agrupamento. Com isso, desenvolveremos uma ferramenta para minerar as informações
disponíveis, realizando os agrupamentos e gerando uma visualização dos programas submetidos.
Para testar a ferramenta, teremos duas bases de dados distintas de implementações. Uma
base de dados existente com soluções de cinco problemas distintos e outra base de dados
com códigos-fontes de alunos. A validação ocorrerá em duas etapas: mediante a qualidade
das visualizações considerando as técnicas de mineração e visualização de dados; e
verificando se o \foreign{feedback} está sendo construtivo ao aluno conforme seu
\foreign{feedback} e da qualidade das próximas implementações.

\textbf{Resultados esperados:} Espera-se que, com a utilização dos subsídios logrados
neste trabalho, potencializar a correção das submissões de modo que, principalmente,
melhore a qualidade do \foreign{feedback} podendo levar menos tempo para corrigi-las
ou o mesmo tempo que a correção tradicional.

% ou, para TCC 2:
% \textbf{Contexto:} \\
% \textbf{Objetivo:} \\
% \textbf{Método:} \\
% \textbf{Resultados:} \\
% \textbf{Conclusões:}

% Palavras-chaves, separadas por ponto (tente não definir mais do que cinco)
\palavraschaves{MOOC. Programação. Agrupamentos. Mineração. Visualização}
\end{resumo}



% Caso seja TCC 2, precisa traduzir o resumo e as palavras-chaves para inglês:
% \begin{abstract}
% \textbf{Context:}
% \textbf{Objective:}
% \textbf{Method:}
% \textbf{Results:}
% \textbf{Conclusions:}

% Palavras-chaves em inglês, separadas por ponto.
% \keywords{}
% \end{abstract}



% Listas (opcionais, mas recomenda-se a partir de 5 elementos)
\listoffigures
\listoftables
\listofacronyms{acronimos}

% Sumário
\tableofcontents

\mainmatter
% TODO: incluir arquivos latex com os capítulos
\chapter{Introdução}

	\ac{MOOC} é uma plataforma de ensino online
	com cursos massivos e abertos que visa oferecer os mais diversos cursos a nível
	global. Para realizar um curso ofertado pelo \acs{MOOC}, é necessário apenas a conexão
	com a Internet e o cadastro na plataforma, geralmente gratuito. Foi popularizado
	em 2011 quando grandes universidades, como Instituto de Tecnologia de
	Massachusetts (MIT), Universidade de Harvard e Universidade de Stanford, tiveram
	a iniciativa mediados pelos provedores Cousera, edX e Udacity, respectivamente
	\cite{Mehlenbacher:2012}. Além de permitir explorar novos modelos de negócio
	\cite{dellarocas2013money}, possibilita englobar mais alunos a custos menores
	e com boa qualidade \cite{schmidt2013producing}.
	
	Em cursos de introdução a programação, seja \acs{MOOC} ou presencial, há uma introdução
	sobre algoritmo seguido da seleção de uma linguagem de programação para
	desenvolvimento. Alguns desses cursos orientam a utilização de ferramentas, como
	o \acs{IDE} e ensinam o funcionamento da linguagem sintaticamente.
	Entretanto, esse tipo de ensino tem levado o estudante a fazer seu programa baseado
	na tentativa e erro, visto que buscam gerar o algoritmo sem o total conhecimento
	lógico do problema \cite{edwards2003}. Isso funcionou como um incentivo para que
	surgissem outras abordagens de ensino, como: baseado em jogos  (\acs{GBL} -- \acl{GBL}) \cite{kapp2012gamification},
	baseado em dicas \cite{glassman2016learnersourcing}, a reflexão na ação (\foreign{Reflection
	in action}) e o \acl{TDD} para ensino (\acs{TDD}) \cite{camara_graciottoSilva2016}.
	A \acs{GBL} utiliza recursos de jogos, como a pontução e o \foreign{ranking}, para motivar
	o estudante, podendo pontuar a cada problema resolvido, por exemplo. A segunda abordagem
	consiste na geração de dicas durante a resolução de um problema para ser apresentado
	a futuros estudantes. Na reflexão na ação, a
	implementação do \foreign{software} ocorre somente após o entendimento apropriado
	do problema \cite{edwards2004}. No \acs{TDD}, cumpre-se um ciclo dividido em três etapas:
	na primeira etapa o programador insere um teste que deve falhar no momento de sua
	execução; na segunda etapa é implementado a solução para que o caso de teste
	escrito anteriormente seja aceito; e a última etapa refere-se a refatoração
	(simplificar ou melhorar) o código recém implementado \cite{beck2003}.
	
	%TODO Criar um gancho de porquê as abordagens de reflexão na ação e TDD são interessantes p/ MOOC
	As abordagens \acl{TDD} e Reflexão na ação tornam-se interessantes para serem
	utilizadas em \acl{MOOC}, visto que ambas necessitam do entendimento do problema
	a ser resolvido, mesmo que de formas diferente. O \acl{TDD} necessita o entendimento
	do que não pode ser feito no problema, com isso, durante sua implementação, é
	possível realizar testes que possam falhar, caso viole alguma regra do problema.
	A Reflexão na ação é interessante devido às etapas impostas nessa abordagem.
	Dessa forma, é preciso entender o problema para que possa realizar as etapas
	corretamente. Ambas as abordagens evitam que os aprendizes tentem simplesmente
	programar e corrigi-los conforme os erros são apresentados. Dessa forma, desenvolvem
	o raciocínio lógico, a interpretação dos problemas e sua implementação.
	
	% Problema de pesquisa
	Principalmente em \acs{MOOC} de ensino de programação, tratando-se de uma ferramenta de
	ensino a nível mundial, deve-se considerar um grande número de usuários. Com isso,
	torna-se necessário a utilização de mecanismos de avaliação automática ou
	semiautomática \cite{schmidt2013producing}. Para alguns tipos de atividade, a
	avaliação automática é bem simples. Por exemplo, para verificar as soluções
	referentes às questões de múltipla escolha é necessário somente comparar a
	alternativa selecionada com o gabarito de questões \cite{alario2013analysing}. No
	entanto, implementações de programas computacionais necessitam que seus algoritmos
	sejam analisados quanto às saídas geradas pela sua execução, projeto do algoritmo,
	facilidade de compreensão, dentro outros requisitos, a fim de retornar o resultado
	da avaliação. Como há uma grande quantidade de submissões de trabalhos em cursos
	de programação, consequentemente acarreta alguns problemas como: avaliar todas as
	submissões, tempo e aumento de custo de correção por parte do professor.
	
	% Hipóteses
	
	% Justificativa
	
	% Objetivo
	Considerando que muitas atividades podem ser semelhantes, os professores podem explorar
	e compreender as variações de implementações enviados pelos estudantes \cite{Yin:2015}, a fim de diminuir
	o tempo gasto para correção dos códigos-fontes submetidos, por meio de agrupamentos
	(\foreign{clusters}). Por exemplo, tais agrupamentos podem ser realizados por meio
	da similaridade dos códigos. Compreendendo as variações de implementação, é possível
	extrair caraterísticas por meio da análise estática \cite{Yin:2015,Glassman:2014,Taherkhani:2012},
	análise dinâmica \cite{Glassman:2015} e análise do estilo de escrita \cite{Wei2015}.
	Com	isso, o agrupamento das submissões semelhantes infere na correção efetiva de
	poucos projetos, já que todos os outros códigos que estão no agrupamento serão
	parecidos, permitindo que o professor dedique o tempo poupado na correção de diversas
	implementações para aprimorar as correções de poucos trabalhos, possibilitando
	\foreign{feedbacks} mais precisos e menor custo para o \acs{MOOC}.
	
	
%	\citeonline{Yin:2015} extraem uma árvore de sintaxe abstrata de cada implementação
%	e verifica a similaridade par a par por meio da distância de edição de árvore. Tal
%	abordagem pode agrupar algoritmos corretos e incorretos, por exemplo: pode ocorrer
%	erro lógico e uma ou mais instruções estarem invertidas, o cálculo de similaridade
%	pode apontar que as implementações são semelhantes e agrupar códigos-fontes corretos
%	com incorretos. Entretanto, essa abordagem possui baixo custo computacional e
%	interessante para possibilidade de adaptação para verificar erros lógicos.
%	
%	\citeonline{Glassman:2014} extraem 60 características de submissões implementadas
%	em Python separadas em duas dimensões: alto nível e baixo nível. Realiza o agrupamento
%	utilizando as características de alto nível e, internamente desses agrupamentos,
%	executa o \foreign{k-means} novamente para as características de baixo nível. Apesar
%	de separar características de alto e baixo nível. Essa abordagem não considera as
%	chamadas de função e recursões, com isso o algoritmo, caso utilize essas instruções,
%	pode não ser classificado. Entretanto, o agrupamento em dois níveis é interessante
%	devido a possibilidade de obter agrupamentos com erros semelhantes apenas das
%	características de nível alto e, após a realização do segundo agrupamento, das
%	características de nível baixo, permitindo a realização do \foreign{feedback} em
%	duas etapas.
%	
%	\citeonline{Taherkhani:2012} apresentam a ferramenta Aari para classificar e reconhecer
%	algoritmos de ordenação. Para isso, extrai diversas características: numéricas,
%	descritivas, outras e de algoritmos de ordenação. Entretanto, é necessário que
%	treine a ferramenta para reconhecer os algoritmos desejados, caso contrário, o
%	classificador não estará apto e ocorrerá diversos erros de classificação. Sua
%	abordagem é interessante pelo fato de extrair características referente a recursão.
%	
%	\citeonline{Glassman:2015} apresentam a ferramenta OverCode que extrai características
%	por meio do histórico de execução da análise dinâmica. A similaridade das submissões
%	é realizada por meio da comparação entre blocos do programa. A partir disso é utilizado
%	o conceito de pilha para apresentar o agrupamentos das implementações semelhantes e,
%	ao clicar em um agrupamento, é possível verificar as linhas de código que formara
%	aquele agrupamento. Essa abordagem proporcionou o desenvolvimento de vários recursos
%	na ferramenta, como reescrever as variáveis que possuem a mesma atribuição, e mostrou
%	que a combinação de blocos de código e análise dinâmica pode produzir bons agrupamentos.
%	
%	\citeonline{Wei2015} realizam a extração de características por meio da normalização
%	do código-fonte e da análise do estilo de escrita. Para isso, separou cada implementação
%	em pedaços de código, condizentes a uma função, para verificar a similaridade entre eles.
%	Com isso, o instrutor corrigiria pedaços do código ao invés de corrigir todo o algoritmo.
%	O autor encontra a menor substring do bloco a partir de comparações \foreign{token} a
%	\foreign{token} para verificar a similaridade. Limitou-se a utilizar os pedaços de
%	código, sendo que poderia ter comparado com os agrupamentos considerando toda a
%	implementação. Entretanto, sua normalização foi interessante e a pouca quantidade de
%	características extraídas pode ter sido primordial para formar os agrupamentos.
	
	O objetivo deste trabalho foi estabelecer subsídios para a avaliação de
	programas submetidos em disciplinas introdutórias à computação, utilizando técnicas
	de mineração e visualização de dados para construir e apresentar agrupamentos de
	código-fonte semelhantes. Os seguintes subsídios foram investigados e desenvolvidos
	para a avaliação de programas submetidos em \acs{MOOC}:  % TODO: a concretização das metas gerarão/são os resultados/subsídios definidos no título do trabalho. Podemos deixar as metas mais claras nesse sentido e colocá-las como itemize.
	\begin{itemize}
		\item Agrupamento dos códigos-fontes semelhantes, utilizando técnicas de mineração;
		\item Seleção de uma técnica de projeção adequado para o mapeamento dos \foreign{clusters};
		\item Ferramenta de visualização;
		\item Material para os professores sobre como corrigir as submissões utilizando a ferramenta.
	\end{itemize}
	
	Como metas, estabeleceu-se
	o desenvolvimento de uma ferramenta para recuperação de dados a partir de
	códigos-fontes e da alteração de uma ferramenta para mineração e visualização de % TODO: apresentar esta ferramenta (Python), como fruto de trabalhos anteriores e utilizadas no presente trabalho
	dados \cite{Alencar-etal:2012}. Para minerar os dados, foi realizado modificações no
	código-fonte do \texttt{Flake8} \cite{flake8} e no \texttt{PEP8} \cite{pep8} para
	que fosse possível armazenar a extração de características em um arquivo. Encontrar
	e modificar essas ferramentas foi fruto de trabalhos anteriores a serem utilizados
	no presente trabalho. Com o auxílio dessas ferramentas, foram investigadas as
	características e técnicas para mineração e visualização dos programas submetidos,
	avaliando-se como contribuir para a correção dos trabalhos submetidos, o tempo em  % TODO: podemos relatar apenas o tempo para correção e a qualidade dos agrupamentos, mas não da qualidade de feedback.
	que foi necessário para que os professores corrigissem todas as submissões e a
	qualidade dos agrupamentos realizados pelo sistema.
	
	% TODO: reescrever este parágrafo para mostrar os resultados do trabalho.
	Por meio desses subsídios buscamos diminuir o tempo gasto na correção de todos os
	códigos-fontes referente às atividades de Introdução a Ciência da Computação e fornecer
	\foreign{feedbacks} construtivos para o professor de modo que ele consiga corrigir
	uma quantidade menor do total de implementações submetidas, produzindo correções
	relevantes para todos os usuários.
	
	O restante desta proposta de trabalho de conclusão de curso é organizado da seguinte
	forma. O \cref{chap:Ref} apresenta o referencial teórico sobre: \acs{MOOC} e visualização,
	além dos trabalhos relacionados à visualização e avaliação de implementações submetidos
	em \acs{MOOC}. O \cref{chap:metodo-ferramentas} apresenta o método de pesquisa, a estratégia
	para avaliação e as ferramentas modificadas e utilizados. O \cref{chap:resultados} descreve
	dois estudos sobre a utilização da ferramenta. O
	\cref{chap:Conclusao} conclui este estudo, apresentando considerações finais e trabalhos
	futuros.

\chapter{Referencial Teórico}
\label{chap:Ref}
	Neste capítulo serão apresentados os conceitos utilizados na pesquisa, bem como
	os estudos que fundamentaram esse projeto. A \cref{sec:Mooc} apresenta
	o conceito de MOOC e as principais características da plataforma de ensino.
	A \cref{sec:FundTeor} descreve as os principais conceitos relacionados à
	mineração e visualização: mineração, visualização, projeção, agrupamento e
	classificador. A \cref{sec:MinVisual} descreve as características que
	podem ser extraídas dado a escolha de um dos tipos de análise de código-fonte.
	Finalmente, a \cref{sec:TrabRel} apresenta as pesquisas relacionados a este estudo
	juntamente com suas limitações.

	\section{Massive Open Online Courses (MOOC)}
	\label{sec:Mooc}
		Após pesquisar o conceito de Curso Massivo, Aberto e \foreign{Online},
		\citeonline{fassbinder2014} observam que não há uma definição comum na
		literatura. Encontram-se três descrições distintas para o termo.
		\citeonline{sivamuni2013} afirmam que o MOOC, em conformidade com o
		dicionário Oxford, é um curso oferecido por meio da Internet, de forma
		gratuita, para uma grande quantidade de alunos. \citeonline{subbian2013}
		reitera que o MOOC disponibiliza curso gratuito, baseado na \foreign{web},
		com registro livre de taxas monetárias e compartilhamento público de
		currículo. Por fim, \citeonline{siemens2013} defende a tendência em inovação e
		experimentação do uso da tecnologia para o ensino a distância e
		\foreign{online} a fim de dar oportunidade de aprendizagem de forma massiva.
		Não obstante, observa-se a concordância quanto ao oferecimento pela Internet
		e para grande quantidade de pessoas (massivo).
		
		\citeonline{kim2014} afirmam que massivo refere-se a capacidade do MOOC em
		suportar uma grande quantidade de alunos. Tal quantidade é bem superior
		ao número de discentes que uma sala pode acomodar ou o total de
		participantes em um curso \foreign{online} antes do surgimento do MOOC.
		Por exemplo, um curso de Inteligência Artificial foi ofertado, \foreign{online}
		e gratuitamente, pela Universidade de Stanford em 2011, no qual houveram
		160.000 estudantes inscritos \cite{rodriguez2012}.
		
		Os principais recursos das plataformas utilizados para MOCC são a integração com outras
		aplicações, como a utilização de e-mails e fóruns, o uso de questionário
		relacionados com os vídeos e a inclusão de atividades para estimular e
		motivar os alunos \cite{fassbinder2014}. Focaremos no desenvolvimento de
		recursos avançados de visualização dos dados obtidos por meio das
		implementações submetidas em cursos de programação e na realização de
		\foreign{feedback} para que o usuário possa verificar seu nível de conhecimento.

	\section{Mineração e Visualização de Dados}
	\label{sec:FundTeor}
		Nesta seção apresentaremos as definições de alguns termos utilizados
		durante a pesquisa: mineração, a diferença na área de Inteligência
		Artificial sobre classificadores e agrupamento, visualização de dados e projeção.

		\begin{figure}[h]
			\centering
			\includegraphics[width=0.6\linewidth]{imagem/mineracaoDados}
			\caption[Etapas da mineração de dados]{Etapas da mineração de dados \cite{rezende2003}}
			\label{fig:mineracaoDados}
		\end{figure}
		
		A mineração de dados é uma etapa do \foreign{Knowledge Discovery in Databases}
		\cite{fayyad1996} que busca descobrir padrões em grandes conjuntos de dados,
		podendo utilizar métodos de inteligência artificial, estatísticos e sistemas
		de banco de dados \cite{chakrabarti2006}. O objeto do processo de mineração
		de dados consiste na extração de informações de um conjunto de dados e
		transformá-lo em uma estrutura compreensível para uso posterior \cite{chakrabarti2006}.
		A \cref{fig:mineracaoDados} apresenta os estágios da mineração de dados:
		o Conhecimento do Domínio condiz com a identificação do problema para possuir
		um conhecimento inicial e definir as metas e os objetivos a serem atingidos
		no processo de extração de conhecimento; a etapa Pré-processamento refere-se
		a padronização e limpeza de dados extraídos de fontes diversas e  a escolha
		de um subconjunto representativo; a fase seguinte, Extração de Padrões, 
		consiste na aplicação do algoritmo de mineração de dados escolhido; o estágio
		Pós-processamento equivale a análise dos padrões obtidos anteriormente e
		possibilita a extração de outros padrões; por fim, a Utilização do Conhecimento
		é a utilização dos dados extraídos em algum sistema ou utilizado diretamente
		pelo usuário.
		
		Muitos métodos de mineração de dados são baseados em técnicas de treinamento
		e teste de aprendizagem de máquina e, reconhecimento de padrões e estatísticas,
		como os algoritmos de classificação e agrupamentos, respectivamente \cite{fayyad1996}.
		
		Classificação é a tarefa de aprendizagem de uma função alvo que mapeia cada
		conjunto de atributos a um dos rótulos de classe predefinidas \cite{Tan:2005:ch4}.
		Uma função alvo auxilia uma ferramenta, que possui informações, a distinguir
		os objetos de diferentes classes. A fim de realizar essas classificações, são
		implementados classificadores com diversas abordagens, como árvores de decisão,
		redes neurais e \foreign{support vector machine}, por exemplo. Esses
		classificadores são mais utilizados para predizer ou descrever um conjunto
		de dados com categorias binárias ou nominais \cite{Tan:2005:ch4}. Por exemplo,
		para classificar um animal como mamífero, réptil, peixe, anfíbio ou pássaro,
		deve-se sumarizar dados como a temperatura do corpo, característica da pele,
		se é uma criatura aquática, se possui patas e hiberna.
		
		O que difere a classificação do agrupamento é que esse é formado por meio da
		comparação de informações entre os objetos e não existem rótulos pré-definidos,
		enquanto aquele é realizado por meio da comparação das informações do objeto
		com os dados contidos na função alvo. Desta forma, os objetos dentro de um grupo
		devem ser similares ou relacionados entre si e diferentes ou não relacionados
		entre objetos de grupos diferentes, ou seja, quanto maior a similaridade dos
		objetos dentro de um grupo e mais diferentes são os agrupamentos, melhor ou
		mais distinto os agrupamentos \cite{Tan:2005:ch8}. O K-means \cite{macqueen1967}
		e o \foreign{Density-based Algorithm for Discovering Clusters} \cite{Ester1996}
		são exemplos de algoritmos de agrupamento.

		Após a mineração dos dados, é desejável a utilização de ferramentas para
		auxiliar na criação de hipóteses sobre conjuntos de dados complexos -- grande
		conjunto de dados ou de alta dimensionalidade -- para que os analistas possuam
		capacidade de explorá-los e compreendê-los \cite{de2003}. A visualização de dados
		produz modelos gráficos e representações visuais a fim de utilizar a capacidade
		cognitiva do ser humano, por meio da percepção visual, para colaborar com
		a exploração e obtenção de informações úteis presente nos dados \cite{de2003,keim2002}.
		Ademais, a visualização de dados é intuitiva, permitindo explorar os dados
		mais rápido e fornecendo melhores resultados na maioria das vezes \cite{keim2002}.
		
		Entretanto, não é possível gerar visualizações sobre conjuntos de dados
		complexos. Assim sendo, é necessário a utilização de técnicas de projeção que
		criam o mapeamento de dados para reconhecimento visual \cite{friedman1974} e,
		com isso, produzir a visualização. Devido a alta dimensionalidade dos dados,
		é necessário a utilização de projeções multidimensionais. Essa técnica realiza
		a diminuição n-dimensional, sendo $n$ uma alta dimensão, para uma espaço
		unidimensional, bidimensional ou tridimensional \cite{paulovich2008least}.
		Para criar essa visualização deve-se selecionar o formato de como as
		características extraídas serão armazenadas, como o vetor de características
		ou o modelo de tabela de dados \cite{de2003}.

	\section{Mineração e Visualização de Programas}
	\label{sec:MinVisual}
		Para realizar a mineração de informações nos programas submetidos, é necessário
		decidir como serão extraídos as características -- análise sintática, dinâmica e
		do código de escrita -- e quais dados podem ser obtidos por meio da análise
		escolhida (\cref{subSec:Caracteristicas}). Independente da dimensão
		obtida por meio da quantidade de informações extraídas, é necessário escolher
		como os dados obtidos serão representados para que seja possível realizar sua visualização.
		
		\subsection{Características de Programas}
		\label{subSec:Caracteristicas}

			A extração de características por meio das implementações podem ocorrer das
			seguintes formas: análise estática, análise do estilo de escrita e análise dinâmica.
			A análise estática ocorre por meio da observação do código-fonte, considerando
			apenas sua implementação, ou seja, não é necessário sua execução. Há diversas
			características que podem ser extraídas dessa análise. Há abordagens que extraem
			somente a Árvore de Sintaxe Abstrata (AST) que pode ser gerada durante a análise
			sintática do compilador para representar o código-fonte em forma de árvore
			armazenando símbolos não-terminais nos nós filhos e símbolos terminais nos
			nós folha, como representa a \cref{fig:AST} em uma declaração de condição.
			Esse tipo de árvore possui símbolos não terminais como nós filhos
			e símbolos terminais como nós folhas. Enquanto outras abordagens extraem
			características como: a quantidade de linhas e atribuições da implementação,
			a complexidade ciclomática \cite{mccabe}, quantidade de variáveis, operadores,
			operandos, laços de repetição e laços de repetição aninhados, por exemplo.
						
			\begin{figure}[h]
				\centering
				\includegraphics[width=0.7\linewidth]{imagem/AST}
				\caption[Representação de árvore de sintaxe abstrata]{Representação de árvore de sintaxe abstrata \cite{louden2004}.}
				\label{fig:AST}
			\end{figure}
			
			A análise do estilo de escrita é um tipo da análise estática. Entretanto,
			se difere no fato das implementações estarem sintaticamente corretas. Nesse
			tipo de análise é considerado o estilo de escrita do programador, abrindo a
			possibilidade de coletar dados como: há mais que uma instrução e importação
			de bibliotecas por linha, há espaços entre operando e operador, os métodos
			são separados por uma linha em branco e tamanho da instrução, medido em
			caracteres. A \cref{tab:exemploEstEsc} exemplifica possíveis características
			extraídas em uma chamada de função, na qual é possível inserir ou não um
			espaço entre o parâmetro e os \foreign{tokens} abrir e fechar parêntese,
			como também a utilização de uma ou mais instruções por linha. Caso a primeira
			chamada de função e as duas primeiras instruções separadas por uma quebra de linha
			forem o padrão para o estilo de escrita, sua ocorrência não gerará nenhum
			aviso. Contudo, caso ocorra as outras três chamadas de função, os quais
			possuem espaços em branco, ou mais de uma instrução por linha separados
			por \texttt{;} (ponto e vírgula), é informado um tipo de erro.
			
			\begin{table}
				\centering
				\begin{tabular}{|c|c|}
					\hline
					Chamada de função & Instrução por linha \\ \hline
					primo(7)		  & a = 5 * 3  \\
					primo( 7)		 & primo(7)	 \\
					primo(7 )		 &	  \\
					primo( 7 )		& a = 5 * 3; primo(7)	\\
					\hline
				\end{tabular}
				\caption[Representação do estilo de escrita]{Representação do estilo
				de escrita em uma chamada de função e instrução por linha.}
				\label{tab:exemploEstEsc}
			\end{table}
			
			A análise dinâmica do código-fonte consiste na observação da execução do
			programa, por meio do \foreign{trace} -- uma espécie de histórico de execução
			do programa -- e de teste de \foreign{software}. Analisando esse histórico é
			possível verificar algumas características, como: em que momento foi realizado
			uma atribuição, chamada de função e recursão, qual bloco de código foi
			executado em uma declaração de condição, a quantidade de vezes que um laço
			de repetição foi executado e a saída no final da execução. Já o teste de
			software pode ser utilizado executando casos de teste nos quais é verificado
			se produção final do programa era o esperado e esta informação, se o caso de
			teste funcionou corretamente ou não, também pode ser uma característica do programa.
			
			Além da possibilidade de extrair os dados citados anteriormente, é possível
			obter os dados de como foi realizado o processo técnico e social do desenvolvimento
			do programa. Para ambas abordagens, é necessário a utilização de um sistema de
			controle de versão para que se possa utilizar seus recursos. Um mecanismo para
			obter informações é em relação ao momento, data e hora em que o aluno realizou
			um \foreign{commit} de uma versão da sua implementação. Outro método é verificar
			se houve comunicações com outras pessoas durante o desenvolvimento do programa,
			por meio da interação em \foreign{issues} ou solicitações de \foreign{pull requests}.
			
			O programa é a implementação do algoritmo que soluciona um problema e é realizado
			mediante seu entendimento. O problema, por sua vez, pode exigir conhecimento e
			uso de técnicas de programação específicas, a utilização de laço de repetição,
			por exemplo. 

	\section{Mineração e visualização}
	\label{sec:TrabRel}
	
		A fim de encontrar a semelhança entre os códigos, \citeonline{Yin:2015}
		utilizaram a AST dos programas submetidos pelos alunos. Após a criação das
		árvores, é necessário o uso de métricas para verificar a similaridade entre
		as árvores. Desta forma, foi utilizada a Distância de Edição de Árvore (TED)
		– ao comparar duas árvores, verificam-se quais são as movimentações (inserção,
		movimentação e remoção) necessárias para que as árvores fiquem iguais. Mais
		precisamente, foi selecionada a TED normalizada, atribuindo pesos maiores para
		os nós mais próximos da raiz (de menor altura) \cite{zhang1989simple}, quanto
		mais próximo do nó raiz, maior sua importância.

		A TED normalizada realiza comparações nó a nó da raiz até as folhas, priorizando
		os nós mais próximos da raiz, atribuindo um peso maior a eles. A fim de evitar
		que pequenas diferenças na sintaxe influenciem diretamente na pontuação de
		similaridade, visto que a estrutura podem ser semelhantes, diferenciando-os
		em detalhes de baixo nível.
		
		Os autores desse artigo utilizaram o algoritmo de agrupamento OPTICS
		\cite{Ankerst1999} para agrupar os códigos semelhantes. Para visualizar os
		agrupamentos foi utilizado o t-SNE \cite{maaten2008} – técnica utilizada para
		reduzir dados de alta dimensionalidade para duas ou três dimensões preservando
		a estrutura local dos dados.
		
		
		\begin{figure}[h]
		\centering
		\includegraphics[width=0.7\linewidth]{imagem/implementacoesYin}
		\caption[Quatro implementações distintas que formaram os agrupamentos]{Quatro implementações distintas que formaram os agrupamentos \cite{Yin:2015}}
		\label{fig:implementacoesYin}
		\end{figure}

		\begin{figure}[ht]
			\centering
			\includegraphics[scale=0.5]{imagem/visualizacao-tSNE.png}
			\caption[Visualização t-SNE]{Visualização t-SNE \cite{Yin:2015}}
			\label{fig:t-SNE}
		\end{figure}

		Na \cref{fig:t-SNE} é possível observar cinco grupos distintos criados a
		partir da comparação das TEDs normalizadas destacados em cores diferentes.
		Tais cores foram obtidas, conforme o foco do trabalho para geração e fornecimento
		de dados, para representar os diversos valores obtidos pela métrica ABC
		\cite{fitzpatrick1997applying}, no qual seu tamanho é computado contando o
		número de atribuições, a quantidade de ramos da árvore e as instruções de
		condição para um segmento do código-fonte. Contudo, esse não é um tópico
		deste projeto. Cada ponto presente na visualização corresponde a uma implementação.
		Todas as implementações possuem a função \foreign{combine\_anagrams} que possui a
		variável \foreign{words} como parâmetro (\cref{fig:implementacoesYin}). A
		implementação em vermelho obteve a menor métrica pela utilização de somente
		uma instrução. Os códigos-fontes roxo e laranja obtiveram pontuação semelhante,
		entretanto, esse obteve uma pontuação maior devido a um laço de repetição para
		cada chave presente na \foreign{hash} antes de terminar a função. Por fim, a
		implementação verde obteve a maior pontuação, visto que utilizou de mais instruções
		para resolver o problema. Enquanto os pontos azuis não obtiveram similaridade
		suficiente para formar um agrupamento ou serem classificados em outro agrupamento,
		ou seja, \foreign{outliers}.

		A extração de características por meio da AST é interessante, como também
		o uso da TED normalizada para verificar sua similaridade, devido ao seu
		baixo custo computacional. Entretanto, como o autor testou seu agrupamento
		somente com implementações para resolver um único problema, não sabemos se
		tal abordagem é eficiente quando houver implementações que buscam resolver
		diversos problemas, em razão da possibilidade dos códigos-fontes gerarem
		árvores parecidas ainda quando solucionam problemas distintos.
		
		Em \citeonline{Glassman:2014}, foi proposto um agrupamento hierárquico de dois
		níveis. No nível mais alto ocorre o agrupamento das soluções, utilizando  o
		\foreign{K-means} com diversos valores para $k$, ao longo do plano de separação,
		considerando apenas características abstratas, como, por exemplo: posição da
		declaração de condicional em relação a declarações de laço de repetição (antes,
		dentro ou depois), profundidade de um laço de repetição (\foreign{loop})
		aninhado, números de nós AST e declarações de retorno, \foreign{loops} e comparações.
		
		Dentro de cada agrupamento de alto nível, é utilizado o \foreign{K-means}
		novamente, para produzir subagrupamentos destinados a capturar a dimensão
		generalizada, construções de linguagem de baixo nível e bibliotecas utilizadas.
		Os agrupamentos internos são formados por meio de 48 características concretas:
		operações aritméticas e lógicas, laços de repetição, funções de bibliotecas,
		declarações de atribuição, \foreign{loops}, condicional, número de variáveis
		e valores constantes, por exemplo.
		
		A validação dos agrupamentos ocorreu por meio da comparação dos \foreign{clusters}
		criados pelo algoritmo de agrupamento com os que foram criados pelos professores.
		Para os professores foram entregues 50 códigos dos estudantes randomicamente
		distribuídos e notou-se que eles ignoraram características de baixo nível.
		Esses autores utilizaram a métrica Informação Mútua Ajustada (AMI, do inglês
		\foreign{Adjusted Mutual Information}), cálculo probabilístico, para comparar
		cada agrupamento dos professores com os \foreign{clusters} gerado pelo \foreign{k-means}.
		Quando o valor de AMI é 0 (zero), quer dizer que os agrupamentos são
		independentes, entretanto, se for igual a 1, indica perfeita concordância
		entre os \foreign{clusters}. Quando $k$ tinha um valor maior ou igual a 15,
		os agrupamentos concordaram com o agrupamento de cada professor, conforme
		medição do AMI.
		
		Apesar da grande quantidade de dados extraídos das implementações, não houve
		nenhum alusão sobre como essas características interfeririam na cálculo de
		similaridade utilizada. Contudo a abordagem de dividir as informações a serem
		coletadas em duas dimensões é relevante. Posto que as implementações com
		a mesma quantidade de laços de repetição, por exemplo, deveriam ser agrupadas
		facilitando a verificação se os alunos entenderam como fazer e utilizar
		tal instrução.
		
		Em \citeonline{Taherkhani:2012}, desenvolveu-se uma ferramenta para classificar
		implementações baseado na utilização de conjuntos de treinamento e teste.
		Com isso, é necessário o conhecimento prévio de implementações dos problemas
		propostos para treinar o classificador e as características para reconhecer as
		submissões dos alunos. Os autores separaram as características em quatro
		categorias: características numéricas, características descritivas,
		características e algoritmos de ordenação e outras características.
		
%		Em \citeonline{Taherkhani:2012}, testou-se a ferramenta Aari para cinco tipos
%		de métodos de ordenação: \foreign{bubble sort}, \foreign{insertion sort},
%		\foreign{selection sort}, \foreign{mergesort} e \foreign{quicksort}. Os autores
%		separaram as características em quatro categorias: características numéricas,
%		características descritivas, características de algoritmos de ordenação e
%		outras características.
		
		A categoria de características numéricas extrai tudo que pode ser medido
		como inteiro e possui o seguinte conjunto de características: número de
		declarações de atribuição; número de linhas de código; complexidade McCabe;
		total de operadores; total de operandos; número de operadores único; número
		de operando único; total do número de operadores e operandos; total do número
		de operadores e operandos únicos; número de variáveis; número de laços de
		repetição; número de laços aninhados e número de bloco.
		
		A categoria de características descritivas possui: se um algoritmo é
		recursivo, se é uma recursão em cauda, papeis (\foreign{roles}) de variáveis
		e \foreign{arrays}. Essas características podem ser identificadas como booleano,
		indicando ausência ou existência das características correspondentes. Enquanto
		outras características possui informações sobre blocos e laços de repetição,
		informação do contador do \foreign{loop} e informações de dependência.
		
		Após extrair as características, cada algoritmo pode ser representado pelo seu
		vetor de características. O Aari, ferramenta de avaliação automática, utiliza
		a técnica de árvore de decisão para classificar os algoritmos. É por meio
		dessa abordagem que os autores classificaram os programas enviados por um
		determinado grupo de alunos.
		
		Para verificar a precisão do Aari, foi realizado uma categorização manual.
		Inicialmente foi realizado um agrupamento manual dos algoritmos de ordenação,
		diferenciando-os em duas etapas. A primeira rodada é referente a implementação
		do algoritmo sem o ensino prévio dos métodos de ordenação descritos
		anteriormente. Desta forma, foi pedido para que 112 alunos implementassem o
		método de ordenação que eles sabiam. Na segunda rodada, foi apresentado
		o funcionamento de cada algoritmo previamente e, após a apresentação, eles
		poderiam implementar qualquer outro algoritmo como também programar o mesmo
		da primeira etapa. Somente 80 alunos participaram da segunda rodada. Esses
		alunos também tinham participado da primeira etapa.
		
		A \cref{fig:clusterManual} apresenta o gráfico do agrupamento manual
		dos métodos de ordenação, \foreign{bubble sort}, \foreign{insertion sort},
		\foreign{selection sort}, \foreign{merge sort} e \foreign{quick sort},
		selecionados para verificar a precisão do Aari. O eixo $x$ é representado
		pelos algoritmos de ordenação citados anteriormente, além da
		\foreign{Inneficiente variations}. Essa última representação é consequência
		de modificações realizadas pelos alunos na estrutura de qualquer algoritmo de
		ordenação. A categoria \foreign{Others} foi criada a partir das implementações
		de outros métodos de ordenação: \foreign{shell sort} e \foreign{heapsort}. O
		eixo $y$ indica a quantidade de implementações reconhecidas. Para cada algoritmo
		de ordenação há duas colunas: a coluna da esquerda representa as soluções
		computacionais da primeira etapa, enquanto a coluna da direita demonstra as
		implementações da segunda rodada. É possível notar, após a apresentação dos
		algoritmos de ordenação na etapa 2, que: poucas implementações foram classificadas
		como \foreign{Inneficiente variations}; menos estudantes optaram em implementar o
		\foreign{bubble sort}, o \foreign{selection sort} e \foreign{Others}; e houve
		mais implementações do \foreign{merge sort} e do \foreign{quick sort}.
		
		\begin{figure}[h]
			\centering
			\includegraphics[scale=0.4]{imagem/clusterManual.png}
			\caption[Agrupamento manual das implementações dos estudos estudantes na
			primeira e segunda etapa]{Agrupamento manual das implementações dos estudos estudantes na
			primeira e segunda etapa \cite{Taherkhani:2012}.}
			\label{fig:clusterManual}
		\end{figure}
		
		Após o agrupamento manual dos métodos de ordenação implementados pelos alunos,
		\citeonline{Taherkhani:2012} realizaram o reconhecimento automático das
		implementações por meio da ferramenta Aari utilizando mais uma categoria de
		características, características de algoritmos de ordenação. Tal categoria
		considera as variáveis mais utilizadas, o uso de variáveis temporárias, se o
		algoritmo necessita de uma memória extra. Caso existam dois \foreign{loops}
		aninhados, pode ocorrer dois tipos de características: o laço externo incrementa
		e o laço interno decrementa; e quando o laço interno é inicializado com o valor
		do laço externo. Inicialmente a ferramenta foi treinada para reconhecer os
		algoritmos de ordenação citados anteriormente. A \cref{fig:clusterAutomatico}
		apresenta um gráfico para comparar cada agrupamento manual realizado anteriormente
		com o reconhecimento automático da ferramenta. Possui as mesmas propriedades da
		\cref{fig:clusterManual} com exceção das colunas: a coluna da esquerda referencia
		o agrupamento manual de cada algoritmo de ordenação da segunda etapa e a coluna
		da direita apresenta os algoritmos reconhecidos corretamente pelo Aari. Nota-se
		que todas as implementações dos métodos de ordenação \foreign{bubble sort},
		\foreign{selection sort} e \foreign{quicksort} foram classificados corretamente.
		É possível verificar também que a ferramenta não foi capaz de reconhecer vários
		algoritmos como \foreign{others}, visto que ele não foi treinado para
		reconhecer tais algoritmos.
		
		\begin{figure}[ht]
			\centering
			\includegraphics[scale=0.33]{imagem/clusterAutomatico.png}
			\caption[Comparação das implementações dos alunos dos \foreign{clusters}
			manual e automático da segunda etapa]{Comparação das implementações dos alunos dos \foreign{clusters}
			manual e automático da segunda etapa \cite{Taherkhani:2012}.}
			\label{fig:clusterAutomatico}
		\end{figure}
		
		A ferramenta mostrou-se capaz de identificar a maioria dos métodos de ordenação
		para quais foi treinada previamente. Tal abordagem torna-se interessante quando
		há um plano de ensino detalhando os problemas a serem resolvidos, possibilitando
		o treinamento da ferramenta. Entretanto, caso seja utilizado para classificar
		diversos problemas desconhecidos para o classificador, não há uma categorização
		prévia adequada. Desta forma, inviabilizaria sua utilização em MOOCs se fosse
		utilizado para classificar os problemas de todos os cursos de programação.
		
		\citeonline{Glassman:2015} apresentam uma ferramenta para visualização de
		implementações que apresenta os agrupamentos de códigos-fontes formados,
		fornecendo as principais instruções utilizadas pelas implementações presentes
		em um determinado conjunto de submissões com a finalidade de auxiliar os
		professores que realizarão as correções.
		
%		\citeonline{Glassman:2015} apresentam o OverCode, ferramenta de visualização
%		de informação que mostra os \foreign{clusters} formados, as principais
%		instruções utilizadas pelas implementações presentes em um determinado
%		agrupamento e as linhas de código de uma determinada função/método. A
%		ferramenta é voltada para aqueles que realizarão a correção das submissões.
		
		Para verificar a similaridade das submissões a fim de realizar o agrupamento,
		são realizados os seguintes passos: formatar o código-fonte, executar casos
		de teste, extrair a sequência de variáveis, identificar variáveis em comum,
		renomear variáveis comuns e únicas para, então, realizar o agrupamento.
		Formatar o código-fonte consiste na reformatação de cada implementação: na
		remoção dos espaços entre os \foreign{tokens} (mantendo os espaços somente
		após as palavras reservadas) de comentários e de linhas em branco. Essas
		modificação no código-fonte, além de deixá-lo legível, também permite cada
		linha de código ser representada como uma \foreign{string} para que seja
		possível encontrá-la em outras soluções \cite{Glassman:2015}.
		
		A segunda etapa do agrupamento consiste na execução do mesmo caso de teste
		para todas as implementações. A cada passo da execução, os nomes e valores de
		variáveis locais e globais, bem como o valor de retorno da função são gravados
		como se fosse um histórico de execução ou \foreign{trace}. A partir desse
		histórico, a ferramenta extrai a sequência de valores de todas as variáveis.
		
		A partir do conhecimento da sequência de valores de cada variável, a ferramenta
		identifica quais são as variáveis comuns. Tais variáveis são reconhecidas a
		partir das suas sequências idênticas dado dois ou mais históricos de execução.
		As variáveis que só ocorrem uma vez no \foreign{trace} são as variáveis únicas.
		Após reconhecer as variáveis comuns e únicas, a ferramenta renomeia-as para o
		nome da variável que ocorreu em mais históricos de execução.
		
		Após todos esses passos, a normalização do código-fonte realizada em todas as
		implementações garante que o estilo de escrita do aluno não influenciará na
		verificação de similaridade. O agrupamento é realizado por meio da comparação entre
		conjuntos de linhas de diversas implementações. Cada agrupamento, representado
		como conjunto de programas, pode possuir 1 ou mais programas cujo conjunto de
		linhas é idêntico.
		
		\begin{figure}[ht]
			\centering
			\includegraphics[scale=0.4]{imagem/overCode.png}
			\caption[Interface da ferramenta OverCode]{Interface da ferramenta OverCode \cite{Glassman:2015}.}
			\label{fig:interfaceOverCode}
		\end{figure}
		
		Na \cref{fig:interfaceOverCode}, que exibe a interface da ferramenta proposta,
		o OverCode, é possível notar a utilização do conjunto de programas para
		representar os agrupamentos. A primeira coluna da esquerda exibe dois painéis.
		O primeiro painel apresenta o número de agrupamentos, representado pelas conjunto de programas,
		e o número total de submissões, enquanto o segundo painel, mostra o maior
		conjunto de programas. A coluna central apresenta a opção de busca, filtrando
		por uma determinada palavra no quadro superior, e as conjunto de programas remanescentes no
		quadro inferior. A terceira coluna apresenta a frequência com que as linhas
		de códigos estão presentes nas soluções dos agrupamentos.

		A comparação dos conjuntos de programas menores com a conjunto de programa maior ocorre entre
		a primeira e a segunda coluna dando ênfase nas linhas que estão implementadas
		diferentes. Com isso, é possível verificar como cada conjunto foi montada e a
		característica daquele agrupamento quando comparada com o conjunto maior. Como
		utiliza análise dinâmica, extraindo o histórico de execução para agrupar as
		submissões, é possível verificar o \foreign{trace} de uma variável ao longo de
		sua execução em um caso de teste a fim de auxiliar os usuários a entenderem a
		execução do algoritmo.
		
		\begin{figure}[h]
			\centering
			\includegraphics[width=1\linewidth]{imagem/OverCodeDistri}
			\caption[Representação do tamanho dos agrupamentos (conjunto de programas) para cada problema]
			{Representação do tamanho dos agrupamentos (conjunto de programas) para cada problema
				\cite{Glassman:2015}}
			\label{fig:OverCodeDistri}
		\end{figure}

		A \cref{fig:OverCodeDistri} apresenta o agrupamento, representado pela quantidade de
		conjunto de programas no eixo $y$ e o total de submissões em uma conjunto de programa no eixo $x$,  em relação a
		três problemas resolvidos em Python: o problema \texttt{iterPower} refere-se a
		implementação de uma função, que possui a base e o expoente como parâmetros, para
		calcular o exponencial com sucessivas multiplicações; o problema \texttt{hangman}
		possui um vetor de caracteres (\foreign{string}) e uma lista de caractere como
		parâmetros, na qual deve retornas todos os caracteres da \foreign{string} que não
		estão presentes na lista de caracteres; por fim, o problema \texttt{compDeriv}
		calcula a derivada de um polinomial, no qual os coeficientes estão presentes em
		uma lista. Esses problemas obtiveram 3875, 1118 e 1433 soluções corretas submetidas,
		respectivamente. Na devida ordem, a maior conjunto de programa de cada problema são constituídas
		de 1534, 97 e 22 implementações e 684, 452 e 959 conjunto de programas com apenas uma solução.

		Por fim, os autores concluíram que a interface auxilia os professores a terem
		uma visão de alto nível das soluções (implementações), podendo compreender os
		erros e fornecer um \foreign{feedback} mais relevante, devido ao agrupamento
		das implementações. Isso diminui consideravelmente a quantidade de submissões
		a serem efetivamente corrigidas.
		
		A análise dinâmica e os recursos realizados para que fosse montado a conjunto de programa
		mostrou-se eficiente, principalmente para o problema \texttt{iterPower}, no
		qual sua maior conjunto de programa obteve quase 40\% das implementações corretas. Com isso,
		torna-se evidente o quanto a ferramenta pode auxiliar os professores a realizar
		as correções dos códigos-fontes. A etapa de padronização do código realizado
		durante a análise, no qual verifica-se as variáveis comuns e renomeia-as,
		contribui com a comparação bloco a bloco, devido a partes do bloco, onde
		ocorrem a utilização dessas variáveis, serem parecidas. Poderia ter sido
		verificado todos os problemas implementados em apenas um código-fonte para
		verificar qual seria o resultado dos agrupamentos e compará-los.
		
		\citeonline{Wei2015} apresentam uma ferramenta para auxiliar na correção das
		submissões dos MOOC's de forma a agrupar pedaços (\foreign{chuncks}) de códigos
		fontes semelhantes, agrupá-los conforme sua similaridade e alocar cada conjunto
		de implementações ao estudante com conhecimento suficiente para revisá-los.
		
		O particionamento do código tem por objetivo torná-lo legível e fácil de
		compreender para a correção. A ferramenta realiza o particionamento por funções.
		Também foi necessário normalizar cada submissão a fim de encontrar o estilo de
		escrita, visto que até mesmo o nome da variável pode alterar o estilo de escrita.
		A ferramenta desenvolvida por \citeonline{Wei2015} normaliza o código por meio
		de três regras: remoção de espaços, linhas em branco e comentários; exclusão
		de palavras reservadas da linguagem de programação, identificadores predefinidos
		e nomes de funções de bibliotecas; e substituição dos identificadores de
		variáveis do usuário por um símbolo especial.
		
		Com isso foi calculado um valor \foreign{hash} para todas as \foreign{k-substring}
		de um código-fonte, sendo $k$ a quantidade de \foreign{substrings} da implementação,
		a fim de verificar a similaridade do estilo de escrita \foreign{token} a
		\foreign{token} e utilizado o algoritmo de \texttt{winnowing} \cite{schleimer2003}
		para escolher o menor subconjunto do estilo de escrita a fim de realizar as
		comparações por meio do coeficiente de similaridade de \foreign{Jaccard} \cite{jaccard1901}.
		
		E com a finalidade de identificar a dificuldade da implementação, foi utilizado
		a distância Euclidiana para comparar as características extraídas -- quantidade
		de métodos invocados e laços de repetição aninhados -- e o \foreign{K Nearest Neighbor}
		(\texttt{k-NN}) para identificar o nível de dificuldade de revisão do
		\foreign{chunck}.
		
		\begin{figure}
			\centering
			\includegraphics[width=0.7\linewidth]{imagem/clusteringPerformance}
			\caption[Distribuição dos agrupamentos]{Distribuição dos agrupamentos \cite{Wei2015}.}
			\label{fig:clusteringPerformance}
		\end{figure}
		
		A \cref{fig:clusteringPerformance} apresenta a distribuição dos
		agrupamentos em quatro problemas distintos. Para todos os gráficos, a
		abcissa é referente ao tamanho dos pedaços, enquanto a ordenada representa
		o número de agrupamentos. É possível notar três tipos de agrupamento. A barra
		horizontal azul representa o agrupamento sem normalização. enquanto as barras
		verde e vermelho possuem o valor de $k$ diferente, entretanto ambas estão
		normalizadas. As tarefas 1 e 2 possuem apenas uma função para resolver um
		pequeno problema e apresentou bons agrupamentos, nos quais possuíam mais
		de 40 implementações em um mesmo grupo quando $k$ era igual a 0,3. As
		tarefas 4 e 5 possuem implementações com várias funções, por esse motivo,
		os agrupamentos não foram satisfeitos, principalmente quando não normalizados.
		Por fim, pode-se notar que a distribuição dos \foreign{chuncks} sem normalização
		de código é inadequado, visto que vários pedaços em todas as tarefas estavam
		sozinhos, e o maior agrupamento possuía entre 6 a 10 funções.
		
		É possível notar que a abordagem dos códigos-fontes compostas de diversas
		funções (tarefas 4 e 5) não obteve sucesso, pois, independente da tarefa
		selecionada, nota-se que a maioria dos \foreign{clusters} formados possuíam
		apenas um código-fonte. Entretanto, a submissão de implementações contendo
		apenas uma função (tarefas 1 e 2) para resolver o problema torna-se interessante,
		devido a realização de alguns agrupamentos com mais de 40 implementações
		semelhantes. Em um ambiente controlado, no qual todos os alunos irão submeter
		diversas soluções com somente uma função, é interessante a utilização da
		ferramenta. Caso contrário, seu uso não interferirá significativamente no
		auxilio a correções de códigos-fontes. 
		
		\begin{landscape}
			\begin{table}[h]
				\tiny
				\begin{tabularx}{\linewidth}{ |X|X|X|X|X|X|X| }
					\hline % Primeira linha
					\textbf{Artigos}
					& \textbf{Dados de entrada}
					& \textbf{Característica extraídas dos dados de entrada}
					& \textbf{Algoritmo para cálculo de distância ou similaridade}
					& \textbf{Algoritmo de clusterização / Classificador}
					& \textbf{Avaliação}
					& \textbf{Conclusões} \\
					\hline % Segunda Linha
					\citeonline{Yin:2015,Moghadam:2015:ATC,choudhury2016autostyle}
					& Implementações
					& Árvore de sintaxe abstrata
					& Distância de Edição de TED normalizada que utiliza estrutura
					\foreign{top-down} e pontuação silhueta
					& OPTICS
					& Clusterização de algoritmos baseado nas soluções de um problema.
					Em cada \foreign{cluster}, identifica as diferenças nas
					implementações de uma abordagem particular
					& TED normalizado demonstra conjuntos mais estáveis e menos discrepantes \\
					\hline % Terceira linha
					\citeonline{Glassman:2014}
					& Implementações
					& Linguagem de alto nível: 12 características - posição de declarações
					condicionais em relação a instruções de \foreign{loop}, profundidade
					dos \foreign{loops} aninhados, números de nós AST, instruções de
					retorno, laços, comparações, etc. Linguagem de baixo nível: 48
					características - operações aritméticas, comparações, \foreign{loops},
					funções de bibliotecas, declarações, número de variáveis do programa,
					valores constantes, etc.
					& Informação Mútua Ajustada
					& K-means
					& Utilizando a métrica AMI comparando os \foreign{cluters} dos feitos
					pelos professores, no qual 0 indica agrupamentos puramente independentes
					e 1, perfeita concordância entre os agrupamentos.
					& Para k maior ou igual a 15, encontrou-se alta concordância entre os
					\foreign{cluster} do k-means e do professor \\
					\hline % Quarta linha
					\citeonline{Taherkhani:2012}
					& Implementações
					& Características numéricas: NAS, LoC, MCC, N1, N2, n1, n2, N, n, Nov,
					NoL, NoNL, NoB. Características descritivas: recursivo, \foreign{tail
					recursive}, funções de variáveis, \foreign{array} auxiliar. Outras
					características: informações de \foreign{loop}/bloco, informações do
					contador de laço, informações de dependência. Características de algoritmos
					de ordenação: MWH, TEMP, In-place, OIID e IITO.
					& Vetor de características,calculada a partir das características extraídas
					das implementações
					& C4.5 (árvore de decisão)
					& Com os algoritmos da primeira rodada, obteve 71\% de precisão; da
					segunda rodada, 81\%.
					& Bom reconhecimento do algoritmo, se estiver conforme a teoria. Taxa de
					acerto considerável baseado em um possível erro do professor. Utilização
					semiautomática: o Aari corrige partes do trabalho no qual foi treinado e
					o professor, o restante \\
					\hline % Quinta linha
					\citeonline{Glassman:2015,glassman2016clustering,glassman2014interacting,glassman2013toward,glassman2013visualizing}
					& Implementações
					& Rastro do programa (Sequência de variáveis, variáveis comuns e
					renomeação de variáveis: colisão comum / comum; colisão de múltiplas
					instâncias; e colisão único/comum), blocos de código.
					& Diferença entre o tamanho dos conjuntos de linhas
					& Comparação entre conjuntos de linhas de código
					& Grande quantidade de conjunto de programas com poucos blocos de código-fonte e poucas
					conjunto de programas com grande quantidade de blocos implementados
					& A partir do OverCode e as conjunto de programas com os algoritmos divididos pelas
					características extraídas, dá a possibilidade de retornar um
					\foreign{feedback} mais preciso para cada grupo e facilita a observação
					da solução do problema. \\
					\hline % Sexta linha
					\citeonline{Wei2015}
					& Implementações
					& Normaliza o código e considera apenas o estilo de escrita do estudante
					& Coeficiente de similaridade de Jaccard
					& Algoritmo \foreign{Winnowing}
					& Classificação de \foreign{workload}: distância Euclidiana e k-NN
					& A agrupamento por pedaços de código aumentou sua eficiência. \\
					\hline
				\end{tabularx}
				\caption{Principais informações dos trabalhos relacionados}
				\label{tab:caracPrinc}
			\end{table}
		\end{landscape}
		
		A \cref{tab:caracPrinc} apresenta as principais características de cada trabalho
		relacionado importantes para o desenvolvimento desse projeto. Possibilitando a
		comparação dos tipos de análises utilizados, quais características foram
		extraídas a partir dessas análises e qual abordagem obteve melhor desempenho.
		
		Ambas as abordagens que utilizaram pedaços das implementações para realizar
		o agrupamento \cite{Glassman:2015,Wei2015} mostraram-se vantajosos pela
		quantidade de conjuntos de linhas ou \foreign{chunks} agrupados. A renomeação
		de variáveis comuns, obtidas por meio de casos de teste, também é interessante
		devido ao fato de igualar as instruções que utilizam tais variáveis a fim de
		realizar as comparações \cite{Glassman:2015}. Apesar da utilização das ASTs
		somente em um único problema \cite{Yin:2015}, tal abordagem obteve agrupamentos
		significativos a fim de auxiliar os professores para realizar as correções.
		
		Contudo, a utilização de um algoritmo de classificação para identificar
		implementações semelhantes \cite{Taherkhani:2012} torna-se inviável a utilização
		da ferramenta em MOOC que contenha diversos tipos de problema, devido a necessidade
		de realizar o conjunto de treinamento e teste a fim de treinar o classificador.
		E a abordagem utilizando dois níveis hierárquicos para agrupamento \cite{Glassman:2014}
		possui um conjunto de características vasto, mas a maioria das características pode
		ser descartada no momento do agrupamento manual realizado pelos professores.

\chapter{Proposta}
\label{chap:Proposta}

	Nesta seção apresentaremos o método utilizado para desenvolvimento do projeto, bem
	como o cronograma a ser seguido para sua finalização.

	\section{Método}
	Após a revisão da literatura, verificamos que o projeto é baseado em algumas etapas.
	Conforme a \cref{fig:fluxogramaProposta}, é necessário escolher qual linguagem de programação
	será utilizado. Em seguida deve-se verificar quais características podem ser extraídas
	das implementações submetidas utilizando a linguagem especificada. Em seguida, é
	necessário construir a base de dados por meio da submissão de implementações. Realizar
	a extração de características. Minerar os dados extraídos das submissões e verificar a
	similaridade entre as implementações. Por fim, realizar a projeção e visualizar os
	agrupamentos.

	\begin{figure}[h]
		\centering
		\includegraphics[width=0.7\linewidth]{imagem/fluxogramaProposta}
		\caption{Fluxograma das etapas necessárias para a realização do projeto}
		\label{fig:fluxogramaProposta}
	\end{figure}

	A \textbf{linguagem de programação} refere-se a definição de qual linguagem será
	utilizada para o desenvolvimento do projeto. Por exemplo: \texttt{C}, \texttt{Java}
	ou \texttt{Python}. Deve ser escolhida conforme a demanda da sua utilização, pela
	disponibilidade de cursos de programação ofertado em MOOCs ou pelo conhecimento de
	ferramentas que possam verificar características dessa linguagem, como o
	\texttt{cpplint} para a linguagem \texttt{C}, \texttt{Checkstyle} para
	\texttt{Java} e \texttt{PEP8} para \texttt{Python}.

	A etapa seguinte \textbf{definir características a serem extraídas} será realizado
	com base nos trabalhos relacionados (\cref{sec:TrabRel}). Verificamos que, é
	possível utilizar características originadas de um tipo de análise (estática,
	dinâmica e do estilo de escrita), bem como associar características de análises
	distintas para obter um melhor resultado.
	
	O estágio \textbf{códigos-fontes} refere-se a construção da base de dados que será
	utilizada no experimento. Tal base tem que possuir uma quantidade considerável de
	submissões de códigos-fontes, devido a quantidade de usuários que utilizam o MOOC,
	para que seja possível avaliar o desenvolvimento do projeto com uma quantidade
	condizente com o número de usuários.
	
	A \textbf{extração de características} é a fase na qual escolhe-se uma ferramenta
	disponibilizada livremente pela Internet que forneça os dados necessários, visto
	que não desejamos criar uma ferramenta para esse fim.

	Enquanto a etapa \textbf{mineração de dados e agrupamento} é responsável pela
	adaptação das ferramentas selecionadas anteriormente para que seja possível obter
	as características e salvá-la em algum tipo de arquivo, como o formato \texttt{XML}
	e o \texttt{CSV}, por exemplo. Em seguida, é necessário a escolha de técnicas que
	seja possível verificar a similaridade dos códigos-fontes a fim de produzir os
	agrupamentos.
	
	Por fim, \textbf{projeção e visualização} refere-se ao emprego de técnicas de
	projeção para que seja possível diminuir a quantidade de dimensões. Uma dimensão
	é ferente a uma característica extraída da implementação, portanto a quantidade
	de dimensões é igual a quantidade de características extraídas. Com isso, é
	necessário selecionar uma técnica para diminuir um espaço $n-dimensional$ para
	tridimensional a fim de realizar a visualização dos agrupamentos.
	
	No caso dos características extraídas não forem relevantes para a formação de
	agrupamentos ou a visualização não produzir modelos gráficos significativos por
	meio dos mapeamento de dados realizados pela técnica de projeção, será necessário
	voltar a segunda etapa para definir outras características a serem extraídas.

	\section{Cronograma}
	
	A \cref{tab:cronograma} apresenta o cronograma para conclusão do Trabalho de
	Conclusão de Curso.
	
	\begin{table}[]
		\begin{tabular}{|l|c|c|c|c|c|c|}
			\hline
			& Janeiro & Fevereiro & Março & Abril & Maio & Junho \\ \hline
			Implementação da \texttt{ProgrammingView} 		& \y  & \y  & \y  & \y  &     &     \\ \hline
			Construção da base de dados      		        &     & \y  & \y  & \y  &     &     \\ \hline
			Extração de caraterísticas         		        &     &     &     &     & \y  &     \\ \hline
			Validação dos agrupamentos      		        &     &     &     &     & \y  & \y  \\ \hline
			Validação dos \foreign{feedbacks}               &     &     &     &     & \y  & \y  \\ \hline
		\end{tabular}
		\caption{Cronograma}
		\label{tab:cronograma}
	\end{table}

\chapter{Resultados Preliminares}
\label{chapterResult}

\bibliographystyle{abntex2-alf}
\bibliography{main} % geração automática das referências a partir do arquivo main.bib

\backmatter
\appendix
\chapter{Erros do PEP8}
\label{apendice:pep8}

	\begin{table}
		\scriptsize
		\begin{tabularx}{\linewidth}{ |l|X|X| }
			\hline
			\textbf{Erro}
			& \textbf{Descrição}
			& \textbf{Exemplo do erro} \\
			\hline
			
			&
			& Correto: if a == 0:$\backslash$n \ \ \ \ a = 1$\backslash$n \ \ \ \ b = 1 \\
			\hline
			E101
			& Indentação contém espaços e tabulações misturados
			& if a == 0:$\backslash$n \ \ \ \ a = 1$\backslash$n$\backslash$tb = 1 \\
			\hline
			
			&
			&  \\
			\hline
			
			&
			& Correto: a = 1 ou if a == 0:$\backslash$n \ \ \ \ a = 1 \\
			\hline
			E111
			& Nível sem indentação e foi encontrado espaços em branco
			&  \\
			\hline
			E112
			& Nível com uma indentação, mas não foi encontrado uma indentação
			&  \\
			\hline
			E113
			& Nível sem indentação e foi encontrado indentação
			&  \\
			\hline
			E114
			& Nível sem indentação e foi encontrado espaços em branco e comentário de uma linha
			&  \\
			\hline
			E115
			& Nível com uma indentação, mas não foi encontrado uma indentação, seguido de comentário
			&  \\
			\hline
			E116
			& Nível sem indentação e foi encontrado indentação seguido de comentário
			&  \\
			\hline
			
			&  
			&  \\
			\hline
			
			&
			& Correto: a = ($\backslash$n) ou a = ($\backslash$n \ \ \ \ 42) \\
			\hline
			E121
			& Há espaços, porém é menos que uma tabulação (4 espaços)
			& a = ($\backslash$n   42) \\
			\hline
			E122
			& Sem indentação
			& a = ($\backslash$n42) \\
			\hline
			E123
			& Não encontra ")" ou indentation of opening bracket's line
			& a = ($\backslash$n \ \ \ \ ) ou a = ($\backslash$n \ \ \ \ 42$\backslash$n \ \ \ \ ) \\
			\hline
			E124
			& Não encontra ")" visualmente indentado
			& a = (24,$\backslash$n \ \ \ \  42$\backslash$n) \\
			\hline
			E125
			& Falta indentação na continuação da declaração
			& if ($\backslash$n \ \ \ \ b):$\backslash$n \ \ \ \ pass \\
			\hline
			E126
			& Excesso de indentação para ")"
			& a = ($\backslash$n \ \ \ \ 42) \\
			\hline
			E127
			& Excesso de indentação afim de indentar visualmente
			& a = (24,$\backslash$n \ \ \ \   42) \\
			\hline
			E128
			& Continuação da linha com limitação de indentação para indentar visualmente
			& a = (24,$\backslash$n \ \ \ \ 42) \\
			\hline
			E129
			& Linha indentada visualmente
			& if (a or$\backslash$n \ \ \ \ b):$\backslash$n \ \ \ \ pass \\
			\hline
			E131
			& unaligned for hanging indent
			& a = ($\backslash$n \ \ \ \ 42$\backslash$n 24) \\
			\hline
			E133
			& Fecha ")" e falta indentação
			&  \\
			\hline
		\end{tabularx}
		\caption{Erros extraídos do estilo de escrita PEP8 referente a indentação}
		\label{tab:pep8E100}
	\end{table}

	\begin{table}
		\tiny
		\begin{tabularx}{\linewidth}{ |l|X|X| }
			\hline
			\textbf{Erro}
			& \textbf{Descrição}
			& \textbf{Exemplo do erro} \\
			\hline
			
			& 
			& Correto: spam(ham[1], \{eggs: 2\}) \\ 
			\hline
			E201 
			& Espaço em branco após (, [ ou \{ 
			& spam( ham[1], \{eggs: 2\}) \\ 
			\hline
			E202 
			& Espaço em branco antes de ), ] ou   
			& spam(ham[1], \{eggs: 2\} ) \\ 
			\hline
			E203 
			& Espaço em branco antes de ":", ";", "," 
			& if x == 4: print x, y; x, y = y , x \\ 
			\hline
			
			& 
			&  \\ 
			\hline
			
			& 
			& Correto: spam(1) ou dict['key'] = list[index] \\ 
			\hline
			E211 
			& Espaço em branco antes de "(" ou "[" 
			& dict ['key'] = list[index] ou dict['key'] = list [index] \\ 
			\hline
			
			& 
			&  \\ 
			\hline
			
			& 
			& Correto: a = 12 + 3 \\ 
			\hline
			E221 
			& Múltiplos espaços antes do operador 
			& a = 4  + 5 \\ 
			\hline
			E222 
			& Múltiplos espaços depois do operador 
			& a = 4 +  5 \\ 
			\hline
			E223 
			& Tabulação antes do operador 
			& a = 4$\backslash$t+ 5 \\ 
			\hline
			E224 
			& Tabulação depois do operador 
			& a = 4 +$\backslash$t5 \\ 
			\hline
			
			& 
			&  \\ 
			\hline
			
			& 
			& Correto: foo(bar, key='word', *args, **kwargs) \\ 
			\hline
			E225 
			& Falta espaço em branco em volta do operador 
			& i=i+1 \\ 
			\hline
			E226 
			& Aritmética 
			& c = (a+b) * (a-b) ou hypot2 = x*x + y*y \\ 
			\hline
			E227 
			& Bit a bit ou shift 
			& c = a|b \\ 
			\hline
			E228 
			& Modulo 
			& msg = fmt\%(errno, errmsg) \\ 
			\hline
			
			& 
			&  \\ 
			\hline
			
			& 
			& Correto: [a, b] ou (3,) ou a[1:4] ou a[1:4:2] \\ 
			\hline
			E231 
			& Não tem espaço em branco após ",", ";" e ":" 
			& ['a','b'] \\ 
			\hline
			
			& 
			&  \\ 
			\hline
			
			& 
			& Correto: a = (1, 2) \\ 
			\hline
			E241 
			& Múltiplos espaços após "," 
			& a = (1,  2) \\ 
			\hline
			E242 
			& Tabulação após "," 
			& a = (1,$\backslash$t2) \\ 
			\hline
			
			& 
			&  \\ 
			\hline
			
			& 
			& Correto: def complex(real, imag=0.0): ou boolean(a >= b) \\ 
			\hline
			E251 
			& Espaços inesperados ao redor de "=" (atribuição) 
			& def complex(real, imag = 0.0): \\ 
			\hline
			
			& 
			&  \\ 
			\hline
			
			& 
			& Correto: x = x + 1  \# Increment x ou x = x + 1 \ \ \ \ \# Increment x \\ 
			\hline
			E261 
			& Menos de dois espaços antes do comentário 
			& x = x + 1 \# Increment x \\ 
			\hline
			E262 
			& Comentário deve iniciar com \# e um espaço 
			& x = x + 1  \#Increment x ou x = x + 1  \#  Increment x \\ 
			\hline
			E265 
			& Comentário em bloco deve inicar com \# e um espaço 
			& \#Block comment \\ 
			\hline
			E266 
			& Muitos \# para o bloco de comentário 
			& \#\#\# Block comment \\ 
			\hline
			
			& 
			&  \\ 
			\hline
			
			& 
			& Correto: True and False \\ 
			\hline
			E271 
			& Múltiplos espaços antes de uma palavra reservada
			& True and  False \\ 
			\hline
			E272 
			& Múltiplos espaços depois de uma palavra reservada 
			& True  and False \\ 
			\hline
			E273 
			& Tabulação após a palavra reservada 
			& True and$\backslash$tFalse \\ 
			\hline
			E274 
			& Tabulação antes da palavra reservada 
			& True$\backslash$tand False \\ 
			\hline
		\end{tabularx}
		\caption{Erros extraídos do estilo de escrita PEP8 referente a espaços em branco}
		\label{tab:pep8E200}
	\end{table}

	\begin{table}
		\scriptsize
		\begin{tabularx}{\linewidth}{ |l|X|X| }
			\hline
			\textbf{Erro}
			& \textbf{Descrição}
			& \textbf{Exemplo do erro} \\
			\hline
			
			& 
			& Correto: def a():$\backslash$n \ \ \ \ pass$\backslash$n$\backslash$n$\backslash$ndef b():$\backslash$n \ \ \ \ pass \\ 
			\hline
			
			& 
			& Correto: def a():$\backslash$n \ \ \ \ pass$\backslash$n$\backslash$n$\backslash$n\# Foo$\backslash$n\# Bar$\backslash$n$\backslash$ndef b():$\backslash$n \ \ \ \ pass \\ 
			\hline
			E301 
			& Espera uma linha em branco e encontrou nenhuma 
			& class Foo:$\backslash$n \ \ \ \ b = 0$\backslash$n \ \ \ \ def bar():$\backslash$n \ \ \ \  \ \ \ \ pass \\ 
			\hline
			E302 
			& Espera duas linhas em branco e encontrou uma quantidade diferente de dois 
			& def a():$\backslash$n \ \ \ \ pass$\backslash$n$\backslash$ndef b(n):$\backslash$n \ \ \ \ pass \\ 
			\hline
			E303 
			& Muitas linhas em branco 
			& def a():$\backslash$n \ \ \ \ pass$\backslash$n$\backslash$n$\backslash$n$\backslash$ndef b(n):$\backslash$n \ \ \ \ pass \\ 
			\hline
			E304 
			& Blank lines found after function decorator - Encontrou linhas em branco após a função principal 
			& @decorator$\backslash$n$\backslash$ndef a():$\backslash$n \ \ \ \ pass \\ 
			\hline
		\end{tabularx}
		\caption{Erros extraídos do estilo de escrita PEP8 referente a linhas em branco}
		\label{tab:pep8E300}
	\end{table}

	\begin{table}
		\scriptsize
		\begin{tabularx}{\linewidth}{ |l|X|X| }
			\hline
			\textbf{Erro}
			& \textbf{Descrição}
			& \textbf{Exemplo do erro} \\
			\hline
			
			& 
			& Correto: import os$\backslash$nimport sys \\ 
			\hline
			E401 
			& Múltiplos import em uma linha 
			& import sys, os \\ 
			\hline
		\end{tabularx}
		\caption{Erros extraídos do estilo de escrita PEP8 referente a importação de bibliotecas}
		\label{tab:pep8E400}
	\end{table}

	\begin{table}
		\scriptsize
		\begin{tabularx}{\linewidth}{ |l|X|X| }
			\hline
			\textbf{Erro}
			& \textbf{Descrição}
			& \textbf{Exemplo do erro} \\
			\hline
			E501 
			& Linha com 80 caractéres ou mais 
			& Correto: aaa = [123,$\backslash$n \ \ \ \ \ \ \ \  \ \ \ \ \ \ \ \ 123] \\
			\hline
			& 
			& Correto: aaa = ("bbb "$\backslash$n \ \ \ \ \ \ \ \  \ \ \ \ \ \ \ \ "ccc") \\
			\hline
			& 
			& Correto: aaa = "bbb " $\backslash$$\backslash$n \ \ \ \ \ \ \ \ "ccc" \\
			\hline
			& 
			& Correto: aaa = 123  \# $\backslash$$\backslash$ \\
			\hline
			& 
			&  \\
			\hline
			E502 
			& Contrabarra é redundante entre parênteses 
			& aaa = [123, $\backslash$$\backslash$n \ \ \ \ \ \ \ \  \ \ \ \ \ \ \ \ 123] ou aaa = ("bbb " $\backslash$$\backslash$n \ \ \ \ \ \ \ \  \ \ \ \ \ \ \ \ "ccc") \\
			\hline
		\end{tabularx}
		\caption{Erros extraídos do estilo de escrita PEP8 referente a tamanho da instrução em caracteres}
		\label{tab:pep8E500}
	\end{table}

	\begin{table}
		\scriptsize
		\begin{tabularx}{\linewidth}{ |l|X|X| }
			\hline
			\textbf{Erro}
			& \textbf{Descrição}
			& \textbf{Exemplo do erro} \\
			\hline
			
			& 
			& Correto: if foo == 'blah':$\backslash$n \ \ \ \ do\_blah\_thing() \\
			\hline
			
			& 
			& Correto: do\_one() ou do\_two() ou do\_three() \\
			\hline
			E701 
			& Várias instruções em uma linha com ":" 
			& if foo == 'blah': do\_blah\_thing() \\
			\hline
			E702 
			& Múltiplas instruções em uma linha separados com ";" 
			& do\_one(); do\_two(); do\_three() \\
			\hline
			E703 
			& Declaração termina com um ";" 
			& do\_four();  \# useless semicolon \\
			\hline
			
			& 
			&  \\
			\hline
			
			& 
			& Correto: if arg is not None: \\
			\hline
			E711 
			& Comparar singleton com com operador lógico diferente
			& if arg != None: \\
			\hline
			E712 
			& Comparar singleton com com operador lógico igual
			& if arg == True: \\
			\hline
			
			& 
			&  \\
			\hline
			
			& 
			& Correto: if x not in y:$\backslash$n \ \ \ \ pass \\
			\hline
			
			& 
			& Correto: assert (X in Y or X is Z) \\
			\hline
			
			& 
			& Correto: if not (X in Y):$\backslash$n    pass \\
			\hline
			
			& 
			& Correto: zz = x is not y \\
			\hline
			E713 
			& Deve-se utilizar not in para verificar se variável está contido em outra 
			& Z = not X in Y ou if not X.B in Y:$\backslash$n \ \ \ \ pass \\
			\hline
			E714 
			& Deve-se utilizar is not para comparar objeto 
			& if not X is Y:$\backslash$n \ \ \ \ pass ou Z = not X.B is Y \\
			\hline
			
			& 
			&  \\
			\hline
			
			& 
			& Correto: if isinstance(obj, int): \\
			\hline
			E721 
			& Não comparar tipos, usar isinstance() 
			& if type(obj) is type(1): \\
			\hline
		\end{tabularx}
		\caption{Erros extraídos do estilo de escrita PEP8 referente a quantidade de instrução por linha e formas de instrução}
		\label{tab:pep8E700}
	\end{table}

	\begin{table}
		\scriptsize
		\begin{tabularx}{\linewidth}{ |l|X|X| }
			\hline
			\textbf{Erro}
			& \textbf{Descrição}
			& \textbf{Exemplo do erro} \\
			\hline
			E901 
			& Checa se a sintaxe é válida 
			&  \\
			\hline
			E902 
			& Tokenize the file, run physical line checks and yield tokens 
			&  \\
			\hline
		\end{tabularx}
		\caption{Erros extraídos do estilo de escrita PEP8 referente a verificação de sintaxe e geração de \foreign{tokens}}
		\label{tab:pep8E900}
	\end{table}

\end{document}
