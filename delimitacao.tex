\chapter{Delimitação}

Esse estudo consiste na utilização de fundamentos das áreas de: Mineração de
dados, Inteligência Artificial e Visualização de Dados. A Mineração de dados
é utilizada por meio do reconhecimento e extração de características do código-fonte.
Após a padronização de um formato de apresentação dessas características, é
necessário o uso de algoritmos não supervisionados de Inteligência Artificial para
que seja realizado o agrupamento das implementações possívelmente semelhantes.
Com isso, emprega-se técnicas de Visualização de Dados para que seja possível
representar os agrupamentos visualmente, por meio de projeções multidimensionais
e algoritmos de visualização.