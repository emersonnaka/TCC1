\chapter{Resultados}
\label{chap:resultados}

	Neste capítulo, são apresentadas as avaliações da projeção e da visualização como
	instrumento para avaliação de programas, ambos considerando um conjunto de
	programas escritos em Python para uma disciplina introdutória à Computação.
	
	
	\section{Descrição da base}
	\label{sec:resultados:base-apoo}

		Essa base é constituída de 152 implementações em Python de 5 problemas distintos.
		Os desenvolvedores das implementações foram anonimizados, sendo representados por 
		sequências de números. Os problemas tratados foram os seguintes. O primeiro exercício
		consiste na revisão de conceitos básicos como: estruturas de condição e laço de repetição. O
		segundo exercício define a criação de classes com mais de um construtor, e manipulação de
		arquivos e cadeia de caracteres. O terceiro exercício consiste na criação das classes
		\texttt{Palavra} e \texttt{Texto}, no qual a primeira classe deve ser utilizada na segunda
		classe, e uma classe para gerar orações gramaticais. O quarto exercício refere-se a
		utilização de herança e polimorfismo, identificando cada palavra contida em um arquivo. O
		quinto exercício consiste na alteração de uma classe, implementando sobreposição de
		operadores. 
		
	
	\section{Avaliação da projeção com preservação de vizinhança}

		A \foreign{ScienceView} utiliza a preservação da vizinhança
		(\cref{subsubsec:vizinhanca}) para verificar a qualidade das projeções. A \cref{fig:neighborhoodAPOO30}
		apresenta a qualidade da projeção para essa base de dados. Essa técnica visa avaliar
		se houve uma preservação da vizinhança dos objetos no espaço de alta dimensionalidade
		na projeção. A \foreign{Neighborhood Preservation} é calculada tomando os $k$ vizinhos
		mais próximos de uma instância no espaço de alta dimensionalidade e os $k$ vizinhos
		mais próximos na projeção, e verificando-se que proporção da vizinhança é preservada
		na projeção. A precisão final para um dado valor de $k$ é a média das precisões para
		cada instância. Esse cálculo é feito para vários valores de $k$ (tipicamente $k=[2,...,30]$)
		de forma a obtermos uma curva. Quanto mais altos os valores da curva para cada valor
		de $k$, maior a qualidade da projeção. Considerando que a técnica \acl{T-LSP} \cite{Alencar}
		gera uma sequência de projeções, temos uma curva para cada projeção da
		sequência.
		
		Por exemplo, na \cref{fig:neighborhoodAPOO30} é possível observar que as projeções no início da
		sequência apresentam maior qualidade quando comparada as demais, como é possível observar
		nas curvas dos problemas 3, 4 e 5. Isso é esperado, dado que, além de serem considerados
		mais programas, os programas referentes aos problemas 3, 4 e 5 são mais complexos do que
		aqueles referentes aos problemas 1 e 2, com características mais diversificadas e difícil
		representação com precisão no espaço 2D da projeção. 
	
		\begin{figure}
			\centering
			\includegraphics[width=0.85\linewidth]{imagem/neighborhoodAPOO30}
			\caption{Qualidade da projeção, utilizando a preservação de vizinhança (\foreign{neighborhood preservation}).}
			\label{fig:neighborhoodAPOO30}
		\end{figure}	
	
		Finalmente, observa-se a precisão de 20\% para a linha (\texttt{year} $5$, obtida com 30 vizinhos para a
		projeção considerando todos os problemas (e, consequentemente, todas as submissões). Considerando
		outros trabalhos que utilizam esta medida para avaliação de qualidade de projeções~\cite{phd:paulovich},
		a qualidade da projeção é adequada e compatível com outras projeções feitas com a técnica LSP e
		similares.
		
		

	
	\section{Avaliação qualitativa da visualização}
	\label{sec:avalQualitativa}
	
		Para obter resultados qualitativos dessa pesquisa, realizamos um estudo experimental com
		as seguintes etapas: treinamento, utilização e questionário.
		
		O treinamento foi realizado na forma de tutorial com a apresentação dos critérios
		de avaliação (estilo de escrita e complexidade ciclomática) e da ferramenta com
		duração de $55$ minutos. Especificamente quanto à ferramenta, foram apresentados
		todos os passos referentes ao seu funcionamento, desde a inserção de uma nova
		coleção no banco de dados até a visualização das implementações com a visualização
		da projeção criada a partir da coleção. Em seguida, foi apresentado como utilizar
		a ferramenta para auxiliar na avaliação das implementações: visualizar os
		agrupamentos formados ao longo das projeções, selecionar um determinado conjunto
		de implementações, identificar as características semelhantes de tais grupos e
		abrir mais de uma implementação simultaneamente. Durante o treinamento, foi utilizada uma
		base de dados distinta da descrita na \cref{sec:resultados:base-apoo}.
		
		% TODO: informar que eles tiveram 20 minutos para avaliar trabalhos. 
		% TODO: falar quantas pessoas foram e como elas foram organizados (dois grupos: primeiramente, um grupo avaliou de forma tradicional, com auxílio dos dados obtidos pela ScienceView-Python e o outro grupo avaliou com auxílio da ferramenta. Depois foram invertidas as intervenções. Em ambos os casos, a cada rodada de avaliações, cada participante criou um relato sobre os programas avaliados e comentários que seriam enviados para os alunos, considerando os erros identificados.)
		Encerrado o treinamento, procedeu-se para a utilização da ferramenta com a base
		descrita na \cref{sec:resultados:base-apoo}. Para iniciarmos essa etapa, foi
		pedido para que os participantes encerrassem o funcionamento da \foreign{ScienceView}
		e a executassem com a criação da base de dados. O experimento teve participação
		de 4 pessoas e ambos os grupos utilizaram os dados obtidos pela \foreign{ScienceView-Python}
		\cref{sec:scienceView-Python}. Todos os contribuintes realizaram  correção manual,
		o qual consistiu na correção somente do código-fonte sem utilizar a ferramenta, e
		somente 3 pessoas utilizaram a ferramenta \foreign{ScienceView} para auxiliar
		na correção. O experimento consistiu em duas etapas, todas envolvendo a correção
		dos programas considerando os tipos de erros considerandos na extração de
		características e a criação de um relato com os programas avaliados e comentários
		de correção. 

		Na primeira etapa, 2 voluntários realizaram a correção manual, utilizando a planilha
		com erros identificados pela ScienceView-Python, enquanto os outros 2
		utilizaram a ferramenta ScienceView para auxiliar na correção. A primeira etapa deve
		duração de 20 minutos. Na segunda etapa, um dos contribuintes que realizou a correção
		manual passou a utilizar a \foreign{ScienceView}\footnote{O outro participante não
		conseguiu executar a ferramenta ScienceView e não participou desta parte do estudo
		experimental.}, enquanto os 2 que utilizaram a ferramenta
		na etapa anterior, realizaram a correção manual durante 20 minutos. Os participantes
		utilizaram a ferramenta sem ajuda no que já tinha sido apresentado e o
		avaliaram por meio de um questionário. 	

		A \cref{fig:projecaoFinal} apresenta a visualização dos agrupamentos das 152
		implementações contidas na base de dados. Isso foi possível após a adaptação da
		ferramenta para leitura de arquivos no formato \texttt{CSV}. Cada ponto da
		visualização é referente a um código-fonte. Ao clicar em um dos pontos,
		uma interface exibe sua implementação e as características extraídas pela
		\foreign{ScienceView-Python} para aquele código-fonte. Também é possível ordenar
		as colunas de características \foreign{Quantity} e \foreign{Normalized} em
		ordem  crescente ou decrescente para visualizar as características que mais
		ocorreram. Por padrão, a tabela é apresentado ordenada de forma decrescente
		pela coluna \foreign{Normalized}.
	
		\begin{figure}[h]
			\centering
			\includegraphics[width=1\linewidth]{imagem/projecaoFinal}
			\caption[Visualização dos agrupamentos da base de dados gerado pela \texttt{ScienceView}]
			{Visualização dos agrupamentos da base de dados gerado pela \texttt{ScienceView}.}
			\label{fig:projecaoFinal}
		\end{figure}
		
		A \cref{tab:resultados} apresenta a quantidade de implementações que cada
		participante corrigiu e de comentários realizados. A primeira linha identifica
		qual participante realizou as correções. E a primeira coluna indica a forma
		de correção realizada pelos voluntários. Então, é apresentado a quantidade
		de correções que cada participante conseguiu realizar e a quantidade de
		comentários conforme o tipo de correção.
		
		\begin{table}[h]
			\small
			\caption{Quantidade de correções e comentários realizados com e sem a utilização da ferramenta}
			\label{tab:resultados}
			\begin{tabularx}{\linewidth}{|X|X|X|X|X|}
		        \hline
		        
		        & Voluntário 1
		        & Voluntário 2
		        & Voluntário 3
		        & Voluntário 4\\
		        
		        \hline
		        Correção tradicional
		        & 3 códigos-fontes e 2 comentários para correção do aluno
		        & 4 códigos-fontes e 3 comentários para correção do aluno
		        & 3 códigos-fontes e 2 comentários para correção do aluno
		        & 5 códigos-fontes e 2 comentários para correção do aluno\\
		        
		        \hline
		        \foreign{ScienceView}
		        & Não utilizou a ferramenta
		        & 4 códigos-fontes e 4 comentários realizados
		        & 3 códigos-fontes e observou que a ferramenta auxiliou no encontro dos erros de forma mais rápida.
		        & 4 códigos-fontes e 3 comentários para correção dos alunos\\
		        \hline
			\end{tabularx}
		\end{table}
		
		Apesar dos 3 voluntários afirmarem que a \foreign{ScienceView} colaborou
		para a correção das implementações, 2 deles corrigiram a mesma quantidade
		de implementações utilizando a ferramenta e manualmente. A exceção ocorreu com 
		apenas 1 participante que extraiu os tópicos semelhantes manualmente. Na
		primeira extração, selecionou apenas 2 códigos-fontes e observou que os
		erros apresentados na tabela de erros eram semelhantes, visto que observou
		a importação de bibliotecas entre funções ao invés de realizá-las no início
		da implementação. Na segunda extração, selecionou 8 códigos-fontes e constatou
		que os erros ocorreram nas implementações e foi compreendido corretamente
		pela ferramenta.


%		As implementações das \cref{fig:codigo1} e \cref{fig:codigo2} foram consideradas
%		semelhantes, devido aos seus respectivos pontos estarem próximos no mapa de
%		projeção. É possível notar que há diversas semelhanças nos tipos das características
%		extraídas e erros que ocorreram pela coluna \foreign{Normalized}, além da quantidade
%		dessas características apresentadas na coluna \foreign{Quantity}.
%		
%		\begin{figure}[h]
%			\centering
%			\includegraphics[width=0.8\linewidth]{imagem/codigo1}
%			\caption[Representação parcial da interface que apresenta o código e suas características]
%			{Representação parcial da interface que apresenta o código e suas características \cite{Alencar-etal:2012}}
%			\label{fig:codigo1}
%		\end{figure}
%		
%		\begin{figure}[H]
%			\centering
%			\includegraphics[width=0.8\linewidth]{imagem/codigo2}
%			\caption[Representação parcial da interface que apresenta o código considerado semelhante ao da \cref{fig:codigo1}]
%			{Representação parcial da interface que apresenta o código considerado semelhante ao da \cref{fig:codigo1} \cite{Alencar-etal:2012}}
%			\label{fig:codigo2}
%		\end{figure}
		
		Para finalizar o estudo experimental, os voluntários responderam um questionário
		anonimamente. O questionário, apresentado no (\cref{apendice:questionario}),
		contém questões
		objetivas e	dissertativas para avaliar a qualidade do treinamento e da ferramenta.
		Enquanto as questões objetivas visavam a qualidade do treinamento, da ferramenta e
		se é possível utilizá-la para auxiliar na correção, as questões dissertativas
		visam a encontrar lacunas observadas pelo participante para uma possível 
		atualização da ferramenta.
		
		A primeira questão refere-se ao tempo de experiência como professor,
		visto que acreditávamos que essa experiência poderia interferir na quantidade
		de correções realizadas com a ferramenta. Dentre eles, $2$ cooperadores possuem
		menos que $1$ ano de experiência, enquanto $1$ possui mais de $10$ anos de
		experiência como professor. Um dos participantes não respondeu o questionário.
		Todos os respondentes afirmaram que o treinamento em forma
		de tutorial da \foreign{ScienceView} contribui para sua utilização.
		
		Obtivemos êxito com a adaptação da interface que apresenta o código-fonte e a
		tabela de erros, visto que, ao serem questionados se faltava
		alguma informação nessa interface, todos responderam negativamente.
		
		Ao relatarem sobre o \foreign{feedback} recebido da \foreign{ScienceView}
		por meio da visualização dos agrupamentos, 2 participantes afirmaram que
		os agrupamentos auxiliaram na verificação de implementações com características
		similares. O outro participantes constatou a possibilidade de utilizar a
		ferramenta em uma turma fechada de alunos a fim de verificar quais os principais
		erros deles por meio da extração de tópicos da ferramenta.
		
		Sobre a experiência em relação ao uso da ferramenta, um dos participantes
		admitiu que, como não utilizou muito a ferramenta, foi mais rápido realizar
		as correções manuais. Contudo, outra resposta afirma que a utilização mais
		frequente da \foreign{ScienceView}, evidenciará analisar as implementações
		de forma mais rápida.
		
		
% TODO: deixar para artigo
%	
%	\section{Projeção e visualização de MIT 6.00.1x}	
%
%	\subsection{Descrição da base}
%	% TODO: descrever base de dados: quais foram os tipos de programas, quantos foram, falar que a
%	% base não foi anonimizada porque os programas estavam publicamente disponíveis no GitHub.
%	Essa base de dados é constituída de 3470 implementações referente a 10 exercícios
%	distintos. Todos as atividades solicitam manipulação de arquivo e cadeia de caracteres.
%	
%	O primeiro exercício requer conceitos de matriz, utilizando lista dentro
%	de lista, e programação dinâmica para solucionar o transporte de animais.
%	
%	O segundo exercício necessita de conhecimento sobre aleatoriedade, lista, condicional,
%	cadeia de caracteres, operações aritméticas e lógicas para implementar o jogo da
%	forca.
%	
%	O terceiro exercício solicita a utilização de laço de repetição, condicional, lista,
%	operações aritméticas e lógicas para implementar o jogo das palavras.
%	
%	O quarto exercício requer conhecimento de lista e dicionário pra codificar e
%	decodificar um texto.
%	O quinto exercício requer o uso de analisador (\foreign{parser}), construção
%	de classe, interface, polimorfismo e operadores lógicos para desenvolver um programa
%	de monitoramento de novos \foreign{feeds} na Internet.
%	
%	O sexto exercício solicita conhecimento sobre criação de classes, matriz, laço de
%	repetição e manipulação de interface gráfica para implementar um aspirador de pó
%	inteligente e sua simulação.
%	
%	O sétimo exercício consiste no uso de classes, aleatoriedade, laço de repetição,
%	condicional, lista e conhecimento de estatística para implementar uma simulação
%	e um sistema de tratamento de pacientes conforme o vírus que eles possuem.
%	
%	O oitavo problema consiste na implementação de classes a partir do exercício $7$.
%	Pedindo a implementação da classe \texttt{ResistantVirus} e \texttt{SimplePatient}
%	para realizar simulações desses vírus em pacientes.
%	
%	O nono exercício requer o uso de dicionário e operador lógico para desenvolver
%	um software que apresente uma lista de assuntos para cada aluno da universidade.
%	
%	E para o décimo exercício, é necessário conhecer um algoritmo de agrupamento para
%	realizar sua implementação.
%	
%	Os desenvolvedores dessas implementações não foram anonimizados, pois seus
%	códigos-fontes estavam presentes em repositórios públicos no GitHub \cite{github}.
%	
%	% TODO: Marco: colocar a string utilizada para buscar os programas e como foi criada a base
%	
%
%\subsection{Avaliação da projeção com preservação de vizinhança}
%
%
%\subsection{Avaliação qualitativa da visualização}
%% TODO: relatar o estudo: treinamento, instruções, questionário, resultados

	\section{Ameaças a validade}
	\label{sec:ameacas}
	
		O treinamento realizado não foi suficiente para avaliar as implementações por meio
		da utilização da \foreign{ScienceView}, visto que um dos participantes relatou que
		a falta de conhecimento sobre a ferramenta fez com que correção manual acontecesse de
		forma mais rápida. Corrobora com isto a opinião de outro participante, que observou que a ferramenta agilizará
		o processo de correção conforme aumentar a frequência de uso da ferramenta. 
		Principalmente, pudemos observar que os participantes do estudo não consideraram a
		avaliação de grupos de trabalho, optando por avaliar os trabalhos individualmente.
		Dadas essas observações, podemos concluir que o treinamento não foi suficiente para
		capacitar todos os participantes sobre a utilização da ferramenta.
		
		Durante o treinamento, não foi exposto explicitamente que, ao utilizar a \foreign{ScienceView},
		a avaliação é quanto aos códigos-fontes. Com isso, além de corrigirem as implementações,
		alguns participantes buscaram verificar se as características semelhantes apontadas
		pela extração de tópicos realmente ocorriam nas implementações. Isso pode ter
		impactado na correção, devido ao tempo gasto para fazer essas verificações.
		
		Outra ameaça é quanto ao tamanho da base de dados.
		Esta era formada por 152 códigos-fontes que solucionam 5 problemas distintos, algo
		considerado pequeno para um MOOC visto que uma das características do \ac{MOOC} é a grande
		quantidade de usuários (\foreign{massive}). Desta forma, essa base de dados não
		reflete ao montante de submissões que podem ocorrer em um \acs{MOOC}
		
		Outra limitação é quanto a quantidade de participantes.
		A aplicação do experimento contou com somente $4$ pessoas, e ainda, somente $3$ pessoas
		utilizarem a \foreign{ScienceView}, não nos retorna nada concreto estatisticamente.
		Para
		viabilizar mais participantes, é necessário, além de tempo, a disponibilidade
		de professores para participar do treinamento e utilizar a ferramenta, focando
		que o desenvolvimento do projeto pode contribuir para avaliação de trabalhos,
		além de possibilitar a verificação do erro mais frequentes a fim de melhorar o
		aprendizado.

	\section{Considerações finais}
	
		Por meio dos resultados obtidos, o próximo capítulo apresenta as conclusões deste trabalho, considerando tanto a análise qualitativa quanto a quantitativa.
