\chapter{Resultados}
\label{chap:resultados}

	% TODO: descrever o conteúdo deste capítulo
	% Neste capítulo, avaliaremos com auxílio da ferramenta um conjunto de programas escritos em
	% Python para uma disciplina introdutória à Computação [..]
	
	
	\section{Descrição da base}
	\label{sec:resultados:base-apoo}

		Essa base é constituída de 152 implementações de 5 problemas distintos. O primeiro exercício
		consiste na revisão de conceitos básicos como: estruturas de condição e laço de repetição. O
		segundo exercício define a criação de classes com mais de um construtor, e manipulação de
		arquivos e cadeia de caracteres. O terceiro exercício consiste na criação das classes
		\texttt{Palavra} e \texttt{Texto}, no qual a primeira classe deve ser utilizada na segunda
		classe, e uma classe para gerar orações gramaticais. O quarto exercício refere-se a
		utilização de herança e polimorfismo, identificando cada palavra contida em um arquivo. E
		o quinto exercício consiste na alteração de uma classe, implementando sobreposição de
		operadores. Os desenvolvedores das implementações foram anonimizados, sendo representado
		por uma sequência de números.
	
	\section{Avaliação da projeção com preservação de vizinhança}

	A \foreign{Science View} (\cref{sec:scienceView}) utiliza a preservação da vizinhança
	(\cref{subsubsec:vizinhanca}) para verificar a qualidade das projeções. A \cref{fig:neighborhoodAPOO30}
	apresenta a qualidade da projeção para essa base de dados. Essa técnica visa avaliar
	se houve uma preservação da vizinhança dos objetos no espaço de alta dimensionalidade
	na projeção. A \foreign{Neighborhood Preservation} é calculada tomando os $k$ vizinhos
	mais próximos de uma instância no espaço de alta dimensionalidade e os $k$ vizinhos
	mais próximos na projeção, e verificando-se que proporção da vizinhança é preservada
	na projeção. A precisão final para um dado valor de $k$ é a média das precisões para
	cada instância. Esse cálculo é feito para vários valores de $k$ (tipicamente $k=[2,...,30]$)
	de forma a obtermos uma curva. Quanto mais baixos os valores da curva para cada valor
	de $k$, maior a qualidade da projeção. Lembrando que na técnica \acl{T-LSP} \cite{Alencar}
	é gerado uma sequência de projeções, e portanto temos uma curva para cada projeção na
	sequência.
	
	Na \cref{fig:neighborhoodAPOO30} é possível observar que as projeções no início da
	sequência apresentam menor qualidade quando comparada as demais, como é possível observar
	nas curvas dos anos $1$ e $2$, nas cores vermelho e azul, respectivamente. Também é
	possível notar que a projeção perde um pouco mais que $12\%$ de informação para as
	projeções dos anos $4$ e $5$ considerando a comparação de $15$ vizinhos do espaço
	$n$-dimensional com o espaço bidimensional.
	
	
	\begin{figure}
		\centering
		\includegraphics[width=0.7\linewidth]{imagem/neighborhoodAPOO30}
		\caption{Qualidade da projeção, utilizando a preservação de vizinhança (\foreign{neighborhood preservation})}
		\label{fig:neighborhoodAPOO30}
	\end{figure}	

	
	\section{Avaliação qualitativa da visualização}
	
		Para obter resultados qualitativos dessa pesquisa, realizamos um estudo experimental com
		as seguintes etapas: treinamento, utilização e questionário. O treinamento foi
		realizado na forma de tutorial com a apresentação dos critérios de avaliação (estilo de
		escrita e complexidade ciclomática) e da ferramenta. Especificamente quanto à ferramenta,
		foram apresentados todos os passos referentes ao seu funcionamento, desde a inserção
		de uma nova coleção no banco de dados até a visualização das implementações com a visualização
		da projeção criada a partir da coleção. Em
		seguida, foi apresentado como utilizar a ferramenta para auxiliar na avaliação das
		implementações: visualizar os
		agrupamentos formados ao longo das projeções, selecionar um determinado conjunto
		de implementações, identificar as características semelhantes de tais grupos e
		abrir mais de uma implementação simultaneamente. Durante o treinamento, foi utilizada uma
		base de dados distinta da descrita na \cref{sec:resultados:base-apoo}.
		
		Encerrado o treinamento, procedeu-se para a utilização da ferramenta com a base descrita na
		\cref{sec:resultados:base-apoo}. Para iniciarmos essa etapa, foi pedido para que os
		participantes encerrassem o funcionamento da \foreign{Science View} e a
		executassem com a criação da base de dados. Os participantes utilizaram a
		ferramenta sem ajuda no que já tinha sido apresentado e o avaliaram por meio  % TODO: informar que eles tiveram 20 minutos para avaliar trabalhos. 
		% TODO: falar quantas pessoas foram e como elas foram organizados (dois grupos: primeiramente, um grupo avaliou de forma tradicional, com auxílio dos dados obtidos pela ScienceView-Python e o outro grupo avaliou com auxílio da ferramenta. Depois foram invertidas as intervenções. Em ambos os casos, a cada rodada de avaliações, cada participante criou um relato sobre os programas avaliados e comentários que seriam enviados para os alunos, considerando os erros identificados.)
		de um questionário.
		
		O questionário (\cref{apendice:questionario}) apresenta questões objetivas e
		dissertativas para avaliar a qualidade do treinamento e da ferramenta. Enquanto
		as questões objetivas visavam a qualidade do treinamento, da ferramenta e se
		é possível utilizá-la para auxiliar na correção, as questões dissertativas
		visam a encontrar lacunas observadas pelo participante para uma possível 
		atualização da ferramenta.
		
		A \cref{fig:projecaoFinal} apresenta a visualização dos agrupamentos das 152
		implementações contidas na base de dados. Isso foi possível após a adaptação da
		ferramenta para leitura de arquivos no formato \texttt{CSV}. Cada ponto da
		visualização é referente a um código-fonte. Ao clicar em um dos pontos, a
		\cref{fig:codigo1} mostra a exibição da sua implementação e as características
		extraídas das ferramentas para aquele código-fonte. Também é possível ordenar
		as colunas de características \foreign{Quantity} e \foreign{Normalized} em
		ordem  crescente ou decrescente para visualizar as características que mais
		ocorreram. Por padrão, a tabela é apresentado ordenada de forma descrescente
		pela coluna \foreign{Normalized}.
	
		\begin{figure}[h]
			\centering
			\includegraphics[width=1\linewidth]{imagem/projecaoFinal} % TODO: substituir, futuramente, pela figura da ferramenta corrigida.
			\caption[Visualização dos agrupamentos da base de dados gerado pela \texttt{Science View}]
			{Visualização dos agrupamentos da base de dados gerado pela \texttt{Science View} \cite{Alencar-etal:2012}}
			\label{fig:projecaoFinal}
		\end{figure}
		
		As implementações das \cref{fig:codigo1} e \cref{fig:codigo2} foram consideradas
		semelhantes, devido aos seus respectivos pontos estarem próximos no mapa de
		projeção. É possível notar que há diversas semelhanças nos tipos das características
		extraídas e erros que ocorreram pela coluna \foreign{Normalized}, além da quantidade
		dessas características apresentadas na coluna \foreign{Quantity}.
		
		\begin{figure}[h]
			\centering
			\includegraphics[width=0.8\linewidth]{imagem/codigo1}
			\caption[Representação parcial da interface que apresenta o código e suas características]
			{Representação parcial da interface que apresenta o código e suas características \cite{Alencar-etal:2012}}
			\label{fig:codigo1}
		\end{figure}
		
		\begin{figure}[H]
			\centering
			\includegraphics[width=0.8\linewidth]{imagem/codigo2}
			\caption[Representação parcial da interface que apresenta o código considerado semelhante ao da \cref{fig:codigo1}]
			{Representação parcial da interface que apresenta o código considerado semelhante ao da \cref{fig:codigo1} \cite{Alencar-etal:2012}}
			\label{fig:codigo2}
		\end{figure}
		
		O experimento teve participação de $3$ pessoas. Todos os contribuintes realizaram
		a correção manual, o qual consistiu na correção somente do código-fonte sem
		utilizar a ferramenta, e utilizaram a ferramenta \foreign{Science View}
		para auxiliar na correção. O experimento consistiu em duas etapas. Na primeira
		etapa, $1$ voluntário realizou a correção manual, enquanto os outros $2$,
		utilizaram a ferramenta para auxiliar na correção com duração de $20$ minutos.
		Na segunda etapa, o contribuinte que realizou a correção manual, passou a
		utilizar a \foreign{Science View}, enquanto os $2$ que utilizaram a ferramenta
		na etapa anterior, realizaram a correção manual durante $22$ minutos.
		
		Para finalizar o experimentos, os voluntários, responderam o questionário (\cref{apendice:questionario})
		anonimamente. A primeira questão refere-se ao tempo de experiência como professor,
		visto que acreditávamos que essa experiência poderia interferir na quantidade
		de correções realizadas com a ferramenta. Dentre eles, $2$ cooperadores possuem
		menos que $1$ ano de experiência, enquanto $1$ possui mais de $10$ anos de
		experiência como professor. E todos eles afirmaram que o treinamento em forma
		de tutorial da \foreign{Science View} contribui para sua utilização.
		
		Obtivemos êxito com a adaptação da interface que apresenta o código-fonte e a
		tabela de erros \cref{fig:codigo1}, visto que, ao serem questionados se faltava
		alguma informação nessa interface, todos responderam negativamente.
		
		Apesar dos $3$ voluntários afirmarem que a \foreign{Science View} colaborou
		para a correção das implementações, $2$ deles corrigiram a mesma quantidade
		de implementações utilizando a ferramenta e manualmente. A exceção ocorreu com 
		apenas $1$ participante que extraiu os tópicos semelhantes manualmente. Na
		primeira extração, selecionou apenas $2$ códigos-fontes e observou que os
		erros apresentados na tabela de erros eram semelhantes, visto que observou
		a importação de bibliotecas entre funções ao invés de realizá-las no início
		da implementação. Na segunda extração, selecionou $8$ códigos-fontes e constatou
		que os erros ocorreram nas implementações e foi compreendido corretamente
		pela ferramenta.
		
		Ao relatarem sobre o \foreign{feedback} recebido da \foreign{Science View}
		por meio da visualização dos agrupamentos, $2$ participantes afirmaram que
		os agrupamentos auxiliou na verificação de implementações com características
		similares. O outro participantes constatou a possibilidade de utilizar a
		ferramenta em uma turma fechada de alunos a fim de verificar quais os principais
		erros deles por meio da extração de tópicos da ferramenta.
		
		Sobre a experiência em relação ao uso da ferramenta, um dos participantes
		admite que, como não utilizou muito a ferramenta, foi mais rápido realizar
		as correções manuais. Contudo, outra resposta afirma que a utilização mais
		frequente da \foreign{Science View}, evidenciará analisar as implementações
		de forma mais rápida.
		
		
% TODO: deixar para artigo
%	
%	\section{Projeção e visualização de MIT 6.00.1x}	
%
%	\subsection{Descrição da base}
%	% TODO: descrever base de dados: quais foram os tipos de programas, quantos foram, falar que a
%	% base não foi anonimizada porque os programas estavam publicamente disponíveis no GitHub.
%	Essa base de dados é constituída de 3470 implementações referente a 10 exercícios
%	distintos. Todos as atividades solicitam manipulação de arquivo e cadeia de caracteres.
%	
%	O primeiro exercício requer conceitos de matriz, utilizando lista dentro
%	de lista, e programação dinâmica para solucionar o transporte de animais.
%	
%	O segundo exercício necessita de conhecimento sobre aleatoriedade, lista, condicional,
%	cadeia de caracteres, operações aritméticas e lógicas para implementar o jogo da
%	forca.
%	
%	O terceiro exercício solicita a utilização de laço de repetição, condicional, lista,
%	operações aritméticas e lógicas para implementar o jogo das palavras.
%	
%	O quarto exercício requer conhecimento de lista e dicionário pra codificar e
%	decodificar um texto.
%	O quinto exercício requer o uso de analisador (\foreign{parser}), construção
%	de classe, interface, polimorfismo e operadores lógicos para desenvolver um programa
%	de monitoramento de novos \foreign{feeds} na Internet.
%	
%	O sexto exercício solicita conhecimento sobre criação de classes, matriz, laço de
%	repetição e manipulação de interface gráfica para implementar um aspirador de pó
%	inteligente e sua simulação.
%	
%	O sétimo exercício consiste no uso de classes, aleatoriedade, laço de repetição,
%	condicional, lista e conhecimento de estatística para implementar uma simulação
%	e um sistema de tratamento de pacientes conforme o vírus que eles possuem.
%	
%	O oitavo problema consiste na implementação de classes a partir do exercício $7$.
%	Pedindo a implementação da classe \texttt{ResistantVirus} e \texttt{SimplePatient}
%	para realizar simulações desses vírus em pacientes.
%	
%	O nono exercício requer o uso de dicionário e operador lógico para desenvolver
%	um software que apresente uma lista de assuntos para cada aluno da universidade.
%	
%	E para o décimo exercício, é necessário conhecer um algoritmo de agrupamento para
%	realizar sua implementação.
%	
%	Os desenvolvedores dessas implementações não foram anonimizados, pois seus
%	códigos-fontes estavam presentes em repositórios públicos no GitHub \cite{github}.
%	
%	% TODO: Marco: colocar a string utilizada para buscar os programas e como foi criada a base
%	
%
%\subsection{Avaliação da projeção com preservação de vizinhança}
%
%
%\subsection{Avaliação qualitativa da visualização}
%% TODO: relatar o estudo: treinamento, instruções, questionário, resultados



	\section{Considerações finais}
	
		O atual banco de dados de implementações, formado por 152 códigos-fontes que
		solucionam 5 problemas distintos é considerado pequeno e pode enviesar o
		projeto. Para isso, construiremos outro banco de dados de implementações
		por meio de submissões de voluntários, visando possuir um conjunto de
		implementações vasto para a pesquisa.
