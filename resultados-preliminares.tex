\chapter{Resultados Preliminares}
\label{chap:Result}

	Na primeira etapa da \cref{fig:fluxogramaProposta}, foi definido que a linguagem
	de programação a ser utilizada será Python. Devido ao conhecimento prévio de
	ferramentas que possam auxiliar na verificação e extração de características,
	bem como a disponibilidade de cursos sobre introdução a programação, utilizando
	Python, em diversos MOOCs.
	
	Considerando a combinação de características originadas de tipos de análises
	distintas, utilizaremos uma combinação de aspectos provenientes da análise
	sintática e do estilo de escrita. A escolha da análise estática se dá a
	possibilidade de implementações para solucionar o mesmo problema possuírem
	o mesmo tamanho. Enquanto a organização do código, realizado pelo programador,
	pode impactar no entendimento da solução do problema, por isso a escolha da
	análise do estilo de escrita. Para isso, utilizaremos a ferramenta \texttt{Flake8}
	\cite{flake8} que pode adquirir todas as características necessárias por meio da
	utilização de \foreign{plugins}.
	
	A etapa \textbf{códigos-fontes} refere-se a construção da base de dados que será
	utilizada no experimento. Inicialmente, possuímos uma pequena base de dados
	constituída de 152 implementações que solucionam 5 problemas distintos. Entretanto,
	esse conjunto de soluções é considerado pequeno e pode enviesar nosso projeto.
	Para isso, construiremos outro banco de dados de implementações por meio de
	submissões de voluntários, visando possuir um conjuntos de códigos-fontes vasto
	para a pesquisa. 
	
	% Quais as características extraídas
	%	Pep8 e McCabe
	No estágio \textbf{extração de características} obteremos as informações necessárias
	por meio da análise estática e do estilo de escrita com o auxílio das ferramentas
	\texttt{PEP8} e \texttt{McCabe} que funcionam como \foreign{plugins} para o
	\texttt{Flake8}. Extrairemos da análise estática somente a quantidade de linhas
	do código-fonte e a complexidade ciclomática de cada \foreign{plugin},
	respectivamente. Contudo, o foco do \texttt{PEP8} é verificar se o estilo de
	escrita PEP 8 \cite{van2001pep} está sendo praticado corretamente. Por esse
	motivo, vamos extrair as características relacionadas a: indentação; espaços em
	branco; linhas em branco; declaração de importação de bibliotecas; tamanho da
	linha; quantidade de instruções por linha e formas de instrução;
	por fim, verificação de sintaxe e geração de \foreign{tokens}.
	
	A indentação possui as seguintes características: tabulações e espaços misturados;
	o nível não possui indentação, mas foi encontrado espaços em branco ou uma indentação,
	ambas podendo ser seguida de um comentário; o nível possui uma indentação, entretanto,
	esse não foi encontrado ou foi encontrado seguido de um comentário; há espaços, contudo
	é menor que uma tabulação de 4 espaços, definido como padrão no \texttt{PEP8}; e em
	instruções de múltiplas linhas verifica se as linhas seguintes estão indentadas com
	a primeira linha da instrução.
	
	Os espaços em branco são analisados em chamadas de função, atribuições, operações
	lógicas e aritméticas, entre palavras reservadas e comentário. Em chamadas de
	funções constata a presença ou falta de espaço em branco depois dos \foreign{tokens}
	\texttt{(}, \texttt{\{} e \texttt{[} ou antes dos \foreign{tokens} \texttt{)},
	\texttt{\}}, \texttt{]}, como também antes de \texttt{(} e \texttt{[} no caso de
	querer acessar um índice de uma lista, por exemplo. Nas atribuições compreende se
	não há espaço em branco entre o operador de atribuição, enquanto nas operações
	lógicas e aritméticas constata se não há espaço entre seus operadores.
	
	% linhas em branco entre métodos e, entre declaração da classe e um método da classe
	
	% importação de bibliotecas
	
	% tamanho da linha
	
	% quantidade de instruções por linha e formas de instrução
	
	% verificação de sintaxe e geração de tokens
	
	% Mineração e similaridade
	% Similaridade do cosseno
	
	% Visualização e projeção
	%	ProgrammingView
	%	Explicar visualização temporal para verificar a qualidade de outras implementações dos alunos