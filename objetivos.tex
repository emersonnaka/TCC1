\chapter{Objetivos}

Considerando que muito projetos podem ser semelhantes, os professores podem explorar
e compreender as variações de implementações \cite{Yin:2015}, a fim de diminuir
o tempo gasto para correção dos códigos-fontes submetidos, por meio de agrupamentos
(\foreign{clusters}). Por exemplo, tais agrupamentos podem ser realizados por meio
da similaridade dos códigos. Compreendendo as variações de implementação, é possível
extrair caraterísticas por meio da análise estática \cite{Yin:2015,Glassman:2014,Taherkhani:2012},
análise dinâmica \cite{Glassman:2015} e análise do estilo de escrita \cite{Wei2015}.
Com	isso, o agrupamento das submissões semelhantes infere na correção efetiva de
poucos projetos, já que todos os outros códigos que estão no agrupamento serão
parecidos, permitindo que o professor dedique o tempo poupado na correção de diversas
implementações para aprimorar as correções de poucos trabalhos, possibilitando
\foreign{feedbacks} mais precisos e menor custo para o \acs{MOOC}.

Considerando este cenário, a concretização dos objetivos deste trabalho gerará
os seguintes subsídios para a avaliação de programas submetidos em \acs{MOOC}:  % TODO: a concretização das metas gerarão/são os resultados/subsídios definidos no título do trabalho. Podemos deixar as metas mais claras nesse sentido e colocá-las como itemize.
\begin{itemize}
	\item Agrupamento dos códigos-fontes semelhantes, utilizando técnicas de mineração;
	\item Técnica de projeção para mapeamento dos \foreign{clusters};
	\item Ferramenta de visualização;
	\item Material para os professores sobre como corrigir as submissões utilizando a ferramenta.
\end{itemize}

Por meio desses subsídios buscaremos diminuir o tempo gasto na correção de todos os códigos
fontes e  fornecer \foreign{feedbacks} construtivos de modo que o usuário consiga
corrigir seus erros e submeta novamente seu algoritmo. Como metas, estabeleceu-se
o desenvolvimento de uma ferramenta para recuperação de dados a partir de
códigos-fontes e da alteração de uma ferramenta para mineração e visualização de
dados \cite{Alencar-etal:2012}. Com o auxílio dessas ferramentas, serão investigadas as
características e técnicas para mineração e visualização dos programas submetidos,
avaliando-se como contribuir para a correção dos trabalhos submetidos, o tempo em
que foi necessário para que os professores corrigissem todas as submissões, a
qualidade dos agrupamentos realizados pelo sistema e, principalmente, a
qualidade do \foreign{feedback}.