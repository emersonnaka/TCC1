\chapter{Objetivos}

O objetivo deste trabalho é estabelecer subsídios para a avaliação de
programas submetidos em disciplinas introdutórias à computação, utilizando técnicas
de mineração e visualização de dados para construir e apresentar agrupamentos de
código-fonte semelhantes. Os seguintes subsídios são investigados e desenvolvidos
para a avaliação de programas submetidos em \acs{MOOC}:

\begin{itemize}
	\item Agrupamento dos códigos-fontes semelhantes, utilizando técnicas de mineração;
	\item Técnica de projeção para mapeamento dos \foreign{clusters};
	\item Ferramenta de visualização;
	\item Material para os professores sobre como corrigir as submissões utilizando a ferramenta.
\end{itemize}

Por meio desses subsídios buscaremos diminuir o tempo gasto na correção de todos os códigos
fontes e  fornecer \foreign{feedbacks} construtivos de modo que o usuário consiga
corrigir seus erros e submeta novamente seu algoritmo. Como metas, estabeleceu-se
o desenvolvimento de uma ferramenta para recuperação de dados a partir de
códigos-fontes e da alteração de uma ferramenta para mineração e visualização de
dados \cite{Alencar-etal:2012}. Com o auxílio dessas ferramentas, serão investigadas as
características e técnicas para mineração e visualização dos programas submetidos,
avaliando-se como contribuir para a correção dos trabalhos submetidos, o tempo em
que foi necessário para que os professores corrigissem todas as submissões, a
qualidade dos agrupamentos realizados pelo sistema e, principalmente, a
qualidade do \foreign{feedback}.