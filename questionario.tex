\chapter{Questionário}
\label{apendice:questionario}

As questões que possuem o símbolo * (asterisco) no final são obrigatórias. As
questões 1, 2, 4, 5, 6, 7, 8 e 9 são objetivas e as questões 3, 10 e 11 são dissertativas.

\begin{enumerate}
	\item Qual seu nível de conhecimento em Python? *\\
	(  ) Básico\\
	(  ) Intermediário\\
	(  ) Avançado
	 	
	\item Qual seu tempo de experiência como professor? *\\
	(  ) Menos que 1 ano\\
	(  ) De 1 a 2 anos\\
	(  ) De 2 a 5 anos\\
	(  ) De 5 a 10 anos\\
	(  ) Acima de 10 anos
	
	\item O tutorial ajudou a utilizar a ferramenta? *\\
	(  ) Sim\\
	(  ) Não
	
	\item Caso a resposta anterior for negativa, quais informações podem ser apresentadas no tutorial?
	
	\item A descrição do exercício estava bem detalhada para solucionar o problema? *\\
	(  ) Sim\\
	(  ) Não
	
	\item A descrição do exercício ocasionou dúvidas em relação ao desenvolvimento da solução? *\\
	(  ) Sim\\
	(  ) Não
	
	\item Faltou alguma informação a ser apresentada na janela que consta o código-fonte? *\\
	(  ) Sim\\
	(  ) Não
	
	\item Em relação a pergunta anterior. Se respondeu ``Sim'', quais informações podem ser apresentadas?
	
	\item A visualização da projeção na Science View, colaborou para a correção das implementações? *\\
	(  ) Sim\\
	(  ) Não
	
	\item Disserte sobre o \foreign{feedback} recebido da Science View por meio da visualização dos agrupamentos. *
	
	\item Disserte sobre a sua experiência em relação ao uso da Science View. *
	
\end{enumerate}