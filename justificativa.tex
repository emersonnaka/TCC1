\chapter{Justificativa}

A conclusão desse estudo pode impactar três envolvidos distintos: o professor;
a instituição que administra o \acs{MOOC}; e o usuário ou aluno. O professor
corrigirá menos submissões, devido ao agrupamento das implementações semelhantes,
além de poder corrigir com maior minuciosidade a fim de melhorar as correções e
realizar melhores \foreign{feedbacks} aos usuários. Pelo fato de corrigir uma
quantidade consideravelmente menor de códigos-fontes, o professor poderá levar
menos tempo para corrigí-las. Com isso, a instituição pagará uma quantidade menor
de dias trabalhados. E o aluno receberá \foreign{feedbacks} bem elaborados,
impactando diretamente na sua aprendizagem.