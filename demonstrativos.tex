\chapter{Demonstrativo da aplicação dos recursos pleiteados}

As técnicas de mineração de dados demandam recursos computacionais avançados para
execução, com requisitos importantes quanto à capacidade de processamento e de
memória (RAM). Além disso, por tratar da visualização de informações, faz-se 
necessário de dispositivo gráfico de alta resolução.

Devido a grande quantidade de informações a serem processadas, é necessário a
utilização de memória RAM com maior frequência, influenciando na quantidade de
transferência de dados da memória. Dessa forma, as especificações que atende
aos requisitos são: capacidade de $4$GB e frequência de $2133$Mhz. Em média,
o custo da memória RAM com essas especificações é de R\$190,00 (duzentos reais).

Para poder visualizar as informações, será necessário a utilização de um dispositivo
gráfico de alta resolução. Com isso, o monitor deve possuir resolução \foreign{Full HD},
entrada de sinal HDMI, além de um painel para visualizar todas as informações contidas
no projeto. Com isso, os modelos com essas especificações valem, em média, R\$810.00
(oitocentos e dez reais).
% http://www.kabum.com.br/cgi-local/site/produtos/descricao_ofertas.cgi?codigo=7790
% http://www.kabum.com.br/cgi-local/site/produtos/descricao_ofertas.cgi?codigo=84405



\chapter{Demonstrativo de vinculação do projeto}

O egresso de Computação possui inserção tanto na indústria quanto na área acadêmica.
No âmbito acadêmico, o problema é relevante e atual, conforme apresentado no
\cref{chapter:problemas-premissas}. Em relação à indústria,
técnicas de mineração de dados são aplicáveis nos mais diversos problemas
de Computação e em aplicações desses na produção e engenharia. Especificamente quanto
aos resultados deste trabalho de conclusão de curso, existe aplicação direta em cursos
dos diferentes níveis educacionais e, em especial, na modalidade de ensino à distância.

