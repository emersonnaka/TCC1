\chapter{Erros do PEP8}
\label{apendice:pep8}

	\begin{table}
		\scriptsize
		\begin{tabularx}{\linewidth}{ |l|X|X| }
			\hline
			\textbf{Erro}
			& \textbf{Descrição}
			& \textbf{Exemplo do erro} \\
			\hline
			
			&
			& Correto: if a == 0:$\backslash$n \ \ \ \ a = 1$\backslash$n \ \ \ \ b = 1 \\
			\hline
			E101
			& Indentação contém espaços e tabulações misturados
			& if a == 0:$\backslash$n \ \ \ \ a = 1$\backslash$n$\backslash$tb = 1 \\
			\hline
			
			&
			&  \\
			\hline
			
			&
			& Correto: a = 1 ou if a == 0:$\backslash$n \ \ \ \ a = 1 \\
			\hline
			E111
			& Nível sem indentação e foi encontrado espaços em branco
			&  \\
			\hline
			E112
			& Nível com uma indentação, mas não foi encontrado uma indentação
			&  \\
			\hline
			E113
			& Nível sem indentação e foi encontrado indentação
			&  \\
			\hline
			E114
			& Nível sem indentação e foi encontrado espaços em branco e comentário de uma linha
			&  \\
			\hline
			E115
			& Nível com uma indentação, mas não foi encontrado uma indentação, seguido de comentário
			&  \\
			\hline
			E116
			& Nível sem indentação e foi encontrado indentação seguido de comentário
			&  \\
			\hline
			
			&  
			&  \\
			\hline
			
			&
			& Correto: a = ($\backslash$n) ou a = ($\backslash$n \ \ \ \ 42) \\
			\hline
			E121
			& Há espaços, porém é menos que uma tabulação (4 espaços)
			& a = ($\backslash$n   42) \\
			\hline
			E122
			& Sem indentação
			& a = ($\backslash$n42) \\
			\hline
			E123
			& Não encontra ")" ou indentation of opening bracket's line
			& a = ($\backslash$n \ \ \ \ ) ou a = ($\backslash$n \ \ \ \ 42$\backslash$n \ \ \ \ ) \\
			\hline
			E124
			& Não encontra ")" visualmente indentado
			& a = (24,$\backslash$n \ \ \ \  42$\backslash$n) \\
			\hline
			E125
			& Falta indentação na continuação da declaração
			& if ($\backslash$n \ \ \ \ b):$\backslash$n \ \ \ \ pass \\
			\hline
			E126
			& Excesso de indentação para ")"
			& a = ($\backslash$n \ \ \ \ 42) \\
			\hline
			E127
			& Excesso de indentação afim de indentar visualmente
			& a = (24,$\backslash$n \ \ \ \   42) \\
			\hline
			E128
			& Continuação da linha com limitação de indentação para indentar visualmente
			& a = (24,$\backslash$n \ \ \ \ 42) \\
			\hline
			E129
			& Linha indentada visualmente
			& if (a or$\backslash$n \ \ \ \ b):$\backslash$n \ \ \ \ pass \\
			\hline
			E131
			& unaligned for hanging indent
			& a = ($\backslash$n \ \ \ \ 42$\backslash$n 24) \\
			\hline
			E133
			& Fecha ")" e falta indentação
			&  \\
			\hline
		\end{tabularx}
		\caption{Erros extraídos do estilo de escrita PEP8 referente a indentação}
		\label{tab:pep8E100}
	\end{table}

	\begin{table}
		\tiny
		\begin{tabularx}{\linewidth}{ |l|X|X| }
			\hline
			\textbf{Erro}
			& \textbf{Descrição}
			& \textbf{Exemplo do erro} \\
			\hline
			
			& 
			& Correto: spam(ham[1], \{eggs: 2\}) \\ 
			\hline
			E201 
			& Espaço em branco após (, [ ou \{ 
			& spam( ham[1], \{eggs: 2\}) \\ 
			\hline
			E202 
			& Espaço em branco antes de ), ] ou   
			& spam(ham[1], \{eggs: 2\} ) \\ 
			\hline
			E203 
			& Espaço em branco antes de ":", ";", "," 
			& if x == 4: print x, y; x, y = y , x \\ 
			\hline
			
			& 
			&  \\ 
			\hline
			
			& 
			& Correto: spam(1) ou dict['key'] = list[index] \\ 
			\hline
			E211 
			& Espaço em branco antes de "(" ou "[" 
			& dict ['key'] = list[index] ou dict['key'] = list [index] \\ 
			\hline
			
			& 
			&  \\ 
			\hline
			
			& 
			& Correto: a = 12 + 3 \\ 
			\hline
			E221 
			& Múltiplos espaços antes do operador 
			& a = 4  + 5 \\ 
			\hline
			E222 
			& Múltiplos espaços depois do operador 
			& a = 4 +  5 \\ 
			\hline
			E223 
			& Tabulação antes do operador 
			& a = 4$\backslash$t+ 5 \\ 
			\hline
			E224 
			& Tabulação depois do operador 
			& a = 4 +$\backslash$t5 \\ 
			\hline
			
			& 
			&  \\ 
			\hline
			
			& 
			& Correto: foo(bar, key='word', *args, **kwargs) \\ 
			\hline
			E225 
			& Falta espaço em branco em volta do operador 
			& i=i+1 \\ 
			\hline
			E226 
			& Aritmética 
			& c = (a+b) * (a-b) ou hypot2 = x*x + y*y \\ 
			\hline
			E227 
			& Bit a bit ou shift 
			& c = a|b \\ 
			\hline
			E228 
			& Modulo 
			& msg = fmt\%(errno, errmsg) \\ 
			\hline
			
			& 
			&  \\ 
			\hline
			
			& 
			& Correto: [a, b] ou (3,) ou a[1:4] ou a[1:4:2] \\ 
			\hline
			E231 
			& Não tem espaço em branco após ",", ";" e ":" 
			& ['a','b'] \\ 
			\hline
			
			& 
			&  \\ 
			\hline
			
			& 
			& Correto: a = (1, 2) \\ 
			\hline
			E241 
			& Múltiplos espaços após "," 
			& a = (1,  2) \\ 
			\hline
			E242 
			& Tabulação após "," 
			& a = (1,$\backslash$t2) \\ 
			\hline
			
			& 
			&  \\ 
			\hline
			
			& 
			& Correto: def complex(real, imag=0.0): ou boolean(a >= b) \\ 
			\hline
			E251 
			& Espaços inesperados ao redor de "=" (atribuição) 
			& def complex(real, imag = 0.0): \\ 
			\hline
			
			& 
			&  \\ 
			\hline
			
			& 
			& Correto: x = x + 1  \# Increment x ou x = x + 1 \ \ \ \ \# Increment x \\ 
			\hline
			E261 
			& Menos de dois espaços antes do comentário 
			& x = x + 1 \# Increment x \\ 
			\hline
			E262 
			& Comentário deve iniciar com \# e um espaço 
			& x = x + 1  \#Increment x ou x = x + 1  \#  Increment x \\ 
			\hline
			E265 
			& Comentário em bloco deve inicar com \# e um espaço 
			& \#Block comment \\ 
			\hline
			E266 
			& Muitos \# para o bloco de comentário 
			& \#\#\# Block comment \\ 
			\hline
			
			& 
			&  \\ 
			\hline
			
			& 
			& Correto: True and False \\ 
			\hline
			E271 
			& Múltiplos espaços antes de uma palavra reservada
			& True and  False \\ 
			\hline
			E272 
			& Múltiplos espaços depois de uma palavra reservada 
			& True  and False \\ 
			\hline
			E273 
			& Tabulação após a palavra reservada 
			& True and$\backslash$tFalse \\ 
			\hline
			E274 
			& Tabulação antes da palavra reservada 
			& True$\backslash$tand False \\ 
			\hline
		\end{tabularx}
		\caption{Erros extraídos do estilo de escrita PEP8 referente a espaços em branco}
		\label{tab:pep8E200}
	\end{table}

	\begin{table}
		\scriptsize
		\begin{tabularx}{\linewidth}{ |l|X|X| }
			\hline
			\textbf{Erro}
			& \textbf{Descrição}
			& \textbf{Exemplo do erro} \\
			\hline
			
			& 
			& Correto: def a():$\backslash$n \ \ \ \ pass$\backslash$n$\backslash$n$\backslash$ndef b():$\backslash$n \ \ \ \ pass \\ 
			\hline
			
			& 
			& Correto: def a():$\backslash$n \ \ \ \ pass$\backslash$n$\backslash$n$\backslash$n\# Foo$\backslash$n\# Bar$\backslash$n$\backslash$ndef b():$\backslash$n \ \ \ \ pass \\ 
			\hline
			E301 
			& Espera uma linha em branco e encontrou nenhuma 
			& class Foo:$\backslash$n \ \ \ \ b = 0$\backslash$n \ \ \ \ def bar():$\backslash$n \ \ \ \  \ \ \ \ pass \\ 
			\hline
			E302 
			& Espera duas linhas em branco e encontrou uma quantidade diferente de dois 
			& def a():$\backslash$n \ \ \ \ pass$\backslash$n$\backslash$ndef b(n):$\backslash$n \ \ \ \ pass \\ 
			\hline
			E303 
			& Muitas linhas em branco 
			& def a():$\backslash$n \ \ \ \ pass$\backslash$n$\backslash$n$\backslash$n$\backslash$ndef b(n):$\backslash$n \ \ \ \ pass \\ 
			\hline
			E304 
			& Blank lines found after function decorator - Encontrou linhas em branco após a função principal 
			& @decorator$\backslash$n$\backslash$ndef a():$\backslash$n \ \ \ \ pass \\ 
			\hline
		\end{tabularx}
		\caption{Erros extraídos do estilo de escrita PEP8 referente a linhas em branco}
		\label{tab:pep8E300}
	\end{table}

	\begin{table}
		\scriptsize
		\begin{tabularx}{\linewidth}{ |l|X|X| }
			\hline
			\textbf{Erro}
			& \textbf{Descrição}
			& \textbf{Exemplo do erro} \\
			\hline
			
			& 
			& Correto: import os$\backslash$nimport sys \\ 
			\hline
			E401 
			& Múltiplos import em uma linha 
			& import sys, os \\ 
			\hline
		\end{tabularx}
		\caption{Erros extraídos do estilo de escrita PEP8 referente a importação de bibliotecas}
		\label{tab:pep8E400}
	\end{table}

	\begin{table}
		\scriptsize
		\begin{tabularx}{\linewidth}{ |l|X|X| }
			\hline
			\textbf{Erro}
			& \textbf{Descrição}
			& \textbf{Exemplo do erro} \\
			\hline
			E501 
			& Linha com 80 caractéres ou mais 
			& Correto: aaa = [123,$\backslash$n \ \ \ \ \ \ \ \  \ \ \ \ \ \ \ \ 123] \\
			\hline
			& 
			& Correto: aaa = ("bbb "$\backslash$n \ \ \ \ \ \ \ \  \ \ \ \ \ \ \ \ "ccc") \\
			\hline
			& 
			& Correto: aaa = "bbb " $\backslash$$\backslash$n \ \ \ \ \ \ \ \ "ccc" \\
			\hline
			& 
			& Correto: aaa = 123  \# $\backslash$$\backslash$ \\
			\hline
			& 
			&  \\
			\hline
			E502 
			& Contrabarra é redundante entre parênteses 
			& aaa = [123, $\backslash$$\backslash$n \ \ \ \ \ \ \ \  \ \ \ \ \ \ \ \ 123] ou aaa = ("bbb " $\backslash$$\backslash$n \ \ \ \ \ \ \ \  \ \ \ \ \ \ \ \ "ccc") \\
			\hline
		\end{tabularx}
		\caption{Erros extraídos do estilo de escrita PEP8 referente a tamanho da instrução em caracteres}
		\label{tab:pep8E500}
	\end{table}

	\begin{table}
		\scriptsize
		\begin{tabularx}{\linewidth}{ |l|X|X| }
			\hline
			\textbf{Erro}
			& \textbf{Descrição}
			& \textbf{Exemplo do erro} \\
			\hline
			
			& 
			& Correto: if foo == 'blah':$\backslash$n \ \ \ \ do\_blah\_thing() \\
			\hline
			
			& 
			& Correto: do\_one() ou do\_two() ou do\_three() \\
			\hline
			E701 
			& Várias instruções em uma linha com ":" 
			& if foo == 'blah': do\_blah\_thing() \\
			\hline
			E702 
			& Múltiplas instruções em uma linha separados com ";" 
			& do\_one(); do\_two(); do\_three() \\
			\hline
			E703 
			& Declaração termina com um ";" 
			& do\_four();  \# useless semicolon \\
			\hline
			
			& 
			&  \\
			\hline
			
			& 
			& Correto: if arg is not None: \\
			\hline
			E711 
			& Comparar singleton com com operador lógico diferente
			& if arg != None: \\
			\hline
			E712 
			& Comparar singleton com com operador lógico igual
			& if arg == True: \\
			\hline
			
			& 
			&  \\
			\hline
			
			& 
			& Correto: if x not in y:$\backslash$n \ \ \ \ pass \\
			\hline
			
			& 
			& Correto: assert (X in Y or X is Z) \\
			\hline
			
			& 
			& Correto: if not (X in Y):$\backslash$n    pass \\
			\hline
			
			& 
			& Correto: zz = x is not y \\
			\hline
			E713 
			& Deve-se utilizar not in para verificar se variável está contido em outra 
			& Z = not X in Y ou if not X.B in Y:$\backslash$n \ \ \ \ pass \\
			\hline
			E714 
			& Deve-se utilizar is not para comparar objeto 
			& if not X is Y:$\backslash$n \ \ \ \ pass ou Z = not X.B is Y \\
			\hline
			
			& 
			&  \\
			\hline
			
			& 
			& Correto: if isinstance(obj, int): \\
			\hline
			E721 
			& Não comparar tipos, usar isinstance() 
			& if type(obj) is type(1): \\
			\hline
		\end{tabularx}
		\caption{Erros extraídos do estilo de escrita PEP8 referente a quantidade de instrução por linha e formas de instrução}
		\label{tab:pep8E700}
	\end{table}

	\begin{table}
		\scriptsize
		\begin{tabularx}{\linewidth}{ |l|X|X| }
			\hline
			\textbf{Erro}
			& \textbf{Descrição}
			& \textbf{Exemplo do erro} \\
			\hline
			E901 
			& Checa se a sintaxe é válida 
			&  \\
			\hline
			E902 
			& Tokenize the file, run physical line checks and yield tokens 
			&  \\
			\hline
		\end{tabularx}
		\caption{Erros extraídos do estilo de escrita PEP8 referente a verificação de sintaxe e geração de \foreign{tokens}}
		\label{tab:pep8E900}
	\end{table}