\chapter{Proposta}
\label{chap:Proposta}

	Nesta seção apresentaremos o método utilizado para desenvolvimento do projeto, bem
	como o cronograma a ser seguido para sua finalização.

	\section{Método}
	Após a revisão da literatura, verificamos que o projeto é baseado em algumas etapas.
	Conforme a \cref{fig:fluxogramaProposta}, é necessário escolher qual linguagem de programação
	será utilizado. Em seguida deve-se verificar quais características podem ser extraídas
	das implementações submetidas utilizando a linguagem especificada. Posteriormente, é
	necessário construir a base de dados por meio da submissão de implementações, realizar
	a extração de características, minerar os dados extraídos das submissões e verificar a
	similaridade entre as implementações. Por fim, realizar a projeção e visualizar os
	agrupamentos.

	\begin{figure}[h]
		\centering
		\includegraphics[width=0.7\linewidth]{imagem/fluxogramaProposta}
		\caption{Fluxograma das etapas necessárias para a realização do projeto}
		\label{fig:fluxogramaProposta}
	\end{figure}

	A \textbf{linguagem de programação} refere-se a definição de qual linguagem será
	utilizada para o desenvolvimento do projeto. Por exemplo: \texttt{C}, \texttt{Java}
	ou \texttt{Python}. Deve ser escolhida conforme a demanda da sua utilização, pela
	disponibilidade de cursos de programação ofertado em MOOCs ou pelo conhecimento de
	ferramentas que possam verificar características dessa linguagem, como o
	\texttt{cpplint} para a linguagem \texttt{C}, \texttt{Checkstyle} para
	\texttt{Java} e \texttt{PEP8} para \texttt{Python}.

	A etapa seguinte \textbf{definir características a serem extraídas} será realizado
	com base nos trabalhos relacionados (\cref{sec:TrabRel}). Verificamos que é
	possível utilizar características originadas de um tipo de análise (estática,
	dinâmica e do estilo de escrita), bem como associar características de análises
	distintas para obter um melhor resultado.
	
	O estágio \textbf{códigos-fontes} refere-se à construção da base de dados que será
	utilizada no experimento. Tal base tem que possuir uma quantidade considerável de
	submissões de códigos-fontes, devido a quantidade de usuários que utilizam o MOOC,
	para que seja possível avaliar o desenvolvimento do projeto com uma quantidade
	condizente com o número de usuários.
	
	A \textbf{extração de características} é a fase na qual escolhe-se uma ferramenta
	disponibilizada livremente pela Internet que forneça os dados necessários, visto
	que não desejamos criar uma ferramenta para esse fim. Tal estágio é responsável pela
	adaptação das ferramentas selecionadas anteriormente para que seja possível obter
	as características e salvá-la em algum tipo de arquivo, como o formato \texttt{XML}
	e o \texttt{CSV}, por exemplo.

	Na etapa \textbf{mineração de dados e agrupamento} analisa-se os dados obtidos a
	fim de verificar um padrão das características obtidas. Isso permite a utilização
	de técnicas para verificar a similaridade dos códigos-fontes a fim
	de produzir os agrupamentos.
	
	Por fim, \textbf{projeção e visualização} refere-se ao emprego de técnicas de
	projeção para que seja possível diminuir a quantidade de dimensões. Uma dimensão
	é referente a uma característica extraída da implementação, portanto a quantidade
	de dimensões é igual a quantidade de características extraídas. Com isso, é
	necessário selecionar uma técnica para diminuir um espaço n-dimensional para
	duas ou três dimensões a fim de realizar a visualização dos agrupamentos.
	
	No caso das características extraídas não forem relevantes para a formação de
	agrupamentos ou a visualização não produzir modelos gráficos significativos por
	meio dos mapeamento de dados realizados pela técnica de projeção, será necessário
	voltar a segunda etapa para definir outras características a serem extraídas.
	
	Nosso objetivo, por meio dessas etapas, consiste no desenvolvimento de subsídios
	de avaliação, utilizando técnicas de visualização para auxiliar os professores a
	corrigirem todas as submissões, considerando o tempo e a qualidade do \foreign{feedback},
	principalmente. Por isso, teremos três questões de pesquisa (QP):
	
	\begin{itemize}
		\item \textbf{QP$_1$}: o algoritmo de agrupamento produz grupos de códigos-fontes
		com boa qualidade?
		\item \textbf{QP$_2$}: a utilização de subsídios de avaliação reduz o tempo
		de correção de todas as submissões?
		\item \textbf{QP$_3$}: a utilização de subsídios de avaliação melhora o \foreign{feedback}?
	\end{itemize}
	
	O subsídio de avaliação será uma adaptação da \texttt{ScienceView} \cite{Alencar-etal:2012}
	que, originalmente, verifica as mudanças nas relações de documentos com o decorrer do tempo,
	utilizando técnicas de mineração, projeção e visualização. Com isso, teremos as
	seguintes hipóteses (HP) em relação a utilização da proposta deste estudo:
	
	\begin{itemize}
		\item \textbf{HP$_1$}: tempo para corrigir os exercícios com os subsídios é inferior
		ao tempo necessário com a correção manual;
		\item \textbf{HP$_2$}: qualidade da resposta (\foreign{feedback}) ao aluno é superior
		àquela com correção manual;
	\end{itemize}
	
	Para validar o tempo gasto e a qualidade dos \texttt{feedbacks} para todas as correções,
	será criado dois grupos de professores. Um grupo utilizará a técnica proposta nessa pesquisa para
	realizar a avaliação das implementações. Enquanto o outro grupo avaliará as submissões
	manualmente. Além disso, a qualidade dos \foreign{feedbacks}, será avaliado através de \foreign{survey}
	e questionário para cada aluno sobre o \foreign{feedback} que recebeu e também será avaliado
	por meio de submissões futuras, visto que a ferramenta realiza várias projeções ao longo do tempo.

	\section{Cronograma}
	
	A \cref{tab:cronograma} apresenta o cronograma para conclusão do Trabalho de
	Conclusão de Curso.
	
	\begin{table}[]
		\begin{tabular}{|l|c|c|c|c|c|c|}
			\hline
			& Janeiro & Fevereiro & Março & Abril & Maio & Junho \\ \hline
			Implementação da \texttt{ScienceView} 		& \y  & \y  & \y  &     &     &     \\ \hline
			Construção da base de dados      		        &     & \y  & \y  &     &     &     \\ \hline
			Extração de caraterísticas         		        &     &     & \y  & \y  &     &     \\ \hline
			Validação dos agrupamentos      		        &     &     &     & \y  & \y  & \y  \\ \hline
			Validação dos \foreign{feedbacks}               &     &     &     & \y  & \y  & \y  \\ \hline
			Publicação dos resultados                       &     &     & \y  & \y  & \y  & \y  \\ \hline
		\end{tabular}
		\caption{Cronograma}
		\label{tab:cronograma}
	\end{table}
