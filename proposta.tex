\chapter{Proposta}
\label{chap:Proposta}

	\section{Método}
	Após a revisão da literatura, verificamos que o projeto é baseado em algumas etapas.
	Conforme a \cref{fig:fluxogramaProposta}, é necessário escolher qual linguagem de programação
	será utilizado. Em seguida deve-se verificar quais características podem ser extraídas
	das implementações submetidas utilizando a linguagem especificada. Em seguida, é
	necessário construir a base de dados por meio da submissão de implementações. Realizar
	a extração de características. Minerar os dados extraídos das submissões e verificar a
	similaridade entre as implementações. Por fim, realizar a projeção e visualizar os
	agrupamentos.

	\begin{figure}[h]
		\centering
		\includegraphics[width=0.7\linewidth]{imagem/fluxogramaProposta}
		\captionsetup{justification=centering}
		\caption{Fluxograma das etapas necessárias para a realização do projeto}
		\label{fig:fluxogramaProposta}
	\end{figure}

	A \textbf{linguagem de programação} refere-se a definição de qual linguagem será
	utilizada para o desenvolvimento do projeto. Por exemplo: \texttt{C}, \texttt{Java}
	ou \texttt{Python}. Deve ser escolhida conforme a demanda da sua utilização, pela
	disponibilidade de cursos de programação ofertado em MOOCs ou pelo conhecimento de
	ferramentas que possam verificar características dessa linguagem, como o
	\texttt{cpplint} para a linguagem \texttt{C}, \texttt{Checkstyle} para
	\texttt{Java} e \texttt{PEP8} para \texttt{Python}.

	A etapa seguinte \textbf{definir características a serem extraídas} será realizado
	com base nos trabalhos relacionados (\cref{sec:TrabRel}). Verificamos que, é
	possível utilizar características originadas de um tipo de análise (estática,
	dinâmica e do estilo de escrita), bem como associar características de análises
	distintas para obter um melhor resultado.
	
	O estágio \textbf{códigos-fontes} refere-se a construção da base de dados que será
	utilizada no experimento. Tal base tem que possuir uma quantidade considerável de
	submissões de códigos-fontes, devido a quantidade de usuários que utilizam o MOOC,
	para que seja possível avaliar o desenvolvimento do projeto com uma quantidade
	condizente com o número de usuários.
	
	A \textbf{extração de características} é a fase na qual escolhe-se uma ferramenta
	disponibilizada livremente pela Internet que forneça os dados necessários, visto
	que não desejamos criar uma ferramenta para esse fim.

	Enquanto a etapa \textbf{mineração de dados e agrupamento} é responsável pela
	adaptação das ferramentas selecionadas anteriormente para que seja possível obter
	as características e salvá-la em algum tipo de arquivo, como o formato \texttt{XML}
	e o \texttt{CSV}, por exemplo. Em seguida, é necessário a escolha de técnicas que
	seja possível verificar a similaridade dos códigos-fontes a fim de produzir os
	agrupamentos.
	
	Por fim, \textbf{projeção e visualização} refere-se ao emprego de técnicas de
	projeção para que seja possível diminuir a quantidade de dimensões. Uma dimensão
	é ferente a uma característica extraída da implementação, portanto a quantidade
	de dimensões é igual a quantidade de características extraídas. Com isso, é
	necessário selecionar uma técnica para diminuir um espaço $n-dimensional$ para
	tridimensional a fim de realizar a visualização dos agrupamentos.
	
	No caso dos características extraídas não forem relevantes para a formação de
	agrupamentos ou a visualização não produzir modelos gráficos significativos por
	meio dos mapeamento de dados realizados pela técnica de projeção, será necessário
	voltar a segunda etapa para definir outras características a serem extraídas.

	\section{Cronograma}