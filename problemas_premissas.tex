\chapter{Problemas e premissas}
\label{chapter:problemas-premissas}

Principalmente em \acs{MOOC} de ensino de programação, tratando-se de uma ferramenta de
ensino a nível mundial, deve-se considerar um grande número de usuários. Com isso,
torna-se necessário a utilização de mecanismos de avaliação automática ou
semiautomática \cite{schmidt2013producing}. Para alguns tipos de atividade, a
avaliação automática é bem simples. Por exemplo, para verificar as soluções
referentes às questões de múltipla escolha é necessário somente comparar a
alternativa selecionada com o gabarito de questões \cite{alario2013analysing}. No
entanto, implementações de programas computacionais necessitam que seus algoritmos
sejam analisados quanto às saídas geradas pela sua execução, projeto do algoritmo,
facilidade de compreensão, dentro outros requisitos, a fim de retornar o resultado
da avaliação. Como há uma grande quantidade de submissões de trabalhos em cursos
de programação, consequentemente acarreta alguns problemas como: avaliar todas as
submissões, tempo e aumento de custo de correção por parte do professor.

Considerando que muito projetos podem ser semelhantes, os professores podem explorar
e compreender as variações de implementações \cite{Yin:2015}, a fim de diminuir
o tempo gasto para correção dos códigos-fontes submetidos, por meio de agrupamentos
(\foreign{clusters}). Por exemplo, tais agrupamentos podem ser realizados por meio
da similaridade dos códigos. Compreendendo as variações de implementação, é possível
extrair caraterísticas por meio da análise estática \cite{Yin:2015,Glassman:2014,Taherkhani:2012},
análise dinâmica \cite{Glassman:2015} e análise do estilo de escrita \cite{Wei2015}.
Com	isso, o agrupamento das submissões semelhantes infere na correção efetiva de
poucos projetos, já que todos os outros códigos que estão no agrupamento serão
parecidos, permitindo que o professor dedique o tempo poupado na correção de diversas
implementações para aprimorar as correções de poucos trabalhos, possibilitando
\foreign{feedbacks} mais precisos e menor custo para o \acs{MOOC}.
