\chapter{Problemas e premissas}
\label{chapter:problemas-premissas}

Principalmente em \acs{MOOC} de ensino de programação, tratando-se de uma ferramenta de
ensino a nível mundial, deve-se considerar um grande número de usuários. Com isso,
torna-se necessário a utilização de mecanismos de avaliação automática ou
semiautomática \cite{schmidt2013producing}. Para alguns tipos de atividade, a
avaliação automática é bem simples. Por exemplo, para verificar as soluções
referentes às questões de múltipla escolha é necessário somente comparar a
alternativa selecionada com o gabarito de questões \cite{alario2013analysing}. No
entanto, implementações de programas computacionais necessitam que seus algoritmos
sejam analisados quanto às saídas geradas pela sua execução, projeto do algoritmo,
facilidade de compreensão, dentro outros requisitos, a fim de retornar o resultado
da avaliação. Como há uma grande quantidade de submissões de trabalhos em cursos
de programação, consequentemente acarreta alguns problemas como: avaliar todas as
submissões, tempo e aumento de custo de correção por parte do professor.

