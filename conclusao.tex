\chapter{Conclusão}
\label{chap:Conclusao}

	Este capítulo apresenta nossa conclusão com base nos resultados obtidos a fim de
	responder nossas questões de pesquisa e os trabalhos futuros para uma possível
	continuidade do projeto.
	
	Por meio dos resultados obtidos, podemos responder nossas questões de pesquisa
	firmadas na \cref{sec:metodo}:
	
	\begin{itemize}
		\item \textbf{QP$_1$}: o algoritmo de agrupamento produz grupos de códigos-fontes
		com boa qualidade?
		\item \textbf{QP$_2$}: a utilização de subsídios de avaliação reduz o tempo
		de correção de todas as submissões?
	\end{itemize}
	
	A fim de responder a \textbf{QP$_1$}, foi estudado e utilizado a preservação
	da vizinhança \cite{paulovich2008hipp}, que quantifica a qualidade de uma projeção.
	Sua utilização ocorreu por meio da \foreign{ScienceView}. Tomaremos, como base
	para a nossa resposta, a \cref{fig:neighborhoodAPOO30}.
	Considerando outros trabalhos que utilizam esta medida para avaliação de qualidade
	de projeções~\cite{phd:paulovich},
	a qualidade da projeção é adequada e compatível com outras projeções feitas com a técnica LSP e similares.
% Por meio dela é possível
%	observar que, quanto maior a quantidade de dados a serem observados, menor a perca
%	na qualidade da projeção.
%	As variações que ocorrem nas curvas vermelha e azul
%	escuro, na instância de tempo $1$ e $2$, respectivamente, possuem perca de
%	qualidade maior quando comparadas com as curvas rosa e azul claro, na instância de
%	tempo $4$ e $5$, respectivamente. Desta forma, responderemos a questão de pesquisa
%	a seguir, utilizando as curvas das instâncias de tempo $4$ e $5$.
%	O algoritmo de agrupamentos gera grupos com boa qualidade, com precisão de $20\%$
%	quando consideramos a quantidade de vizinhos
%	igual a $30$. No melhor caso, ocorre uma perca de aproximadamente $12,5\%$ quando
%	a quantidade de vizinhos é igual a $15$. Desta forma, entre $15$ vizinhos de um
%	código-fonte qualquer, no máximo duas implementações estão agrupadas inadequadamente. 
	
	Para responder a \textbf{QP$_2$} foi necessário aplicar o estudo experimental
	para saber quantas implementações os participantes analisariam em um determinado
	período de tempo, realizando as correções manualmente e por meio da utilização dos
	subsídios proposto. A utilização de subsídios não reduziu o tempo de correção
	das implementações em um período de $20$ minutos, visto que alguns participantes
	corrigiram a mesma quantidade de implementações nas duas formas de análise, enquanto
	um participante realizou mais correções manualmente, quando comparado a utilização
	da ferramenta.

	Contudo, quando utilizaram a\foreign{ScienceView}, podemos observar
	que os participantes realizaram poucas correções de implementações que estão no
	mesmo agrupamento. O Voluntário 2 (\cref{tab:resultados}) analisou uma implementação
	de um agrupamento com 9 códigos-fontes, duas implementações de um grupo com 12
	códigos-fontes e uma implementação de um agrupamento contendo 5 códigos-fontes
	distintos. O Voluntário 3, apesar de corrigir uma implementação que estava isolada
	na projeção, analisou uma implementação de um agrupamento contendo 12 códigos-fontes
	e uma implementação de um agrupamento com 8 códigos-fontes. Finalmente, o Voluntário 4
	corrigiu uma implementação que estava isolada na projeção, uma implementação de
	um agrupamento contendo 5 códigos-fontes, e duas implementação de agrupamentos
	distintos contendo 2 códigos-fontes distintos. Apesar de não podermos comprovar
	a segunda questão de pesquisa, o total de códigos-fontes que podem ser considerados
	corrigidos por meio da correção de uma implementação do agrupamento é alta. Isso
	nos dá indícios que, se melhorarmos	o treinamento, poderemos alcançar claramente
	nosso objetivo.
	
%	A negativa para essa resposta pode ser pela pouca experiência de uso
%	com a \foreign{ScienceView}, como também a preocupação de alguns participantes em
%	verificar se os erros apresentados na extração de tópicos realmente aconteceram
%	nas implementações.
	
	Como trabalhos futuros, é importante sanar as ameaças a validade (\cref{sec:ameacas})
	desse projeto. Focando, inicialmente, na construção de uma base de dados de
	implementações maior, a fim de tornar-se próximo a quantidade de submissões
	possíveis em um \acs{MOOC}. Em seguida, melhorar nosso treinamento para que
	todos os participantes utilizem a ferramenta para sua finalidade, e encontrar
	mais professores dispostos a participar de todo o experimento.
	
	Para uma base de dados maior, seria interessante avaliar o quanto as características
	extraídas neste estudo são relevantes e, caso necessário, realimentar o
	sistema da mineração de dados, como descrito na \cref{sec:FundTeor}. Por
	exemplo, poderia-se considerar características dinâmicas das submissões, executando
	casos de teste e, encontrando as variáveis
	comuns e renomeando-as para o nome mais utilizado \cite{Glassman:2015}.
	A combinação da análise dinâmica e estática pode gerar resultados melhores e
	particularmente úteis para avaliação de programas.
