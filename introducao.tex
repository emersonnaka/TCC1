\chapter{Introdução}
% Contextualização
Com o aumento da acessibilidade ao ensino superior, muitos profissionais estão em
igualdade perante o mercado de trabalho, com isso há uma procura de outros meios
para qualificação a fim de obter conhecimento intelectual para o crescimento
pessoal e profissional, por consequência. Desta forma, foi desenvolvido métodos
de ensino online para que qualquer pessoa possa fazê-lo.

MOOC - \textit{Massive Open Online Course} - é uma plataforma de ensino online
que visa oferecer os mais diversos cursos a nível global \cite{Mehlenbacher:2012},
visto que, para realizar um curso ofertado pelo MOOC, é necessário apenas a conexão
com a internet. Essa plataforma é uma forma de ensino a distância, desta forma,
os usuários possuem a liberdade de planejar seus horários que realizará as aulas. 

% Problema de pesquisa
Tratando-se de uma ferramenta de ensino a nível mundial, deve-se considerar um
grande número de usuários. Portanto, há uma grande quantidade de submissões de
trabalhos em cursos de programação, visto que, a cada seção de um curso, existem
atividades a serem feitas e entregues por meio da plataforma de ensino. Desta
forma, há uma grande quantidade de projetos a serem corrigidos e, consequentemente,
acarreta alguns problemas como: avaliar todas as submissões, tempo e aumento de
custo de correção por parte do professor e do assistente de ensino
(\textit{teaching assistant} – TA).

% Hipóteses


% Objetivo
Considerando que muito projetos podem ser semelhantes, os professores e os TAs
podem explorar e compreender as variações de implementações \cite{Yin:2015}, a
fim de diminuir o tempo gasto para correção dos códigos fontes submetidos, por
meio de agrupamentos (\textit{clusters}). Os agrupamentos serão realizados por
meio da similaridade dos códigos a partir de métricas de software e agrupados
com um algoritmo de classificação.

Considerando este cenário, o objetivo deste trabalho é propor subsídios para a
avaliação de programas submetidos, utilizando técnicas de mineração e visualização
de dados com a finalidade de diminuir o tempo gasto na correção de todos os códigos
fontes e  fornecer \textit{feedbacks} expressivos. Especificamente, serão
onsiderados cursos introdutórios de Computação com as linguagens Python. Como
metas, estabeleceu-se o desenvolvimento de uma ferramenta para recuperação de
dados a partir de códigos fonte em Python e da alteração de uma ferramenta para
mineração e visualização de dados~\cite{Alencar}. Com o auxílio dessas ferramentas,
serão investigadas as características e técnicas para mineração e visualização
dos programas submetidos, avaliando-se como o quanto ele pode contribuir para
a correção dos códigos fontes submetidos, o tempo em que foi necessário para
que os assistentes corrigirem todas as submissões e a qualidade dos agrupamentos
realizados pelo sistema.

% Justificativa
Corrigir cada submissão individualmente, além de cansar o TA, gera um custo maior
para a organização responsável pelo MOOC, visto que, quanto o maior o tempo gasto
para correção das implementações, maior será o valor pago ao profissional. Com
isso, o agrupamento das submissões semelhantes infere na correção de poucos
projetos, já que todos os outros códigos que estão no agrupamento serão
parecidos, podendo melhorar a qualidade de vida do TA, a eficiência da correção
por meio de \textit{feedbacks} mais precisos e menor custo para o MOOC.

Após o desenvolvimento do sistema, avaliaremos o quanto ele pode contribuir para
a correção dos códigos fontes submetidos, o tempo em que foi necessário para que
os assistentes corrigissem todas as submissões e a qualidade dos agrupamentos
realizados pelo sistema.