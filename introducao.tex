\chapter{Introdução}

	\ac{MOOC} é uma plataforma de ensino online
	com cursos massivos e abertos que visa oferecer os mais diversos cursos a nível
	global. Para realizar um curso ofertado pelo \acs{MOOC}, é necessário apenas a conexão
	com a Internet e o cadastro na plataforma, geralmente gratuito. Foi popularizado
	em 2011 quando grandes universidades, como o Instituto de Tecnologia de
	Massachusetts, a Universidade de Harvard e a Universidade de Stanford tiveram
	a iniciativa mediados pelos provedores Cousera, edX e Udacity, respectivamente
	\cite{Mehlenbacher:2012}. Além de permitir explorar novos modelos de negócio
	\cite{dellarocas2013money}, possibilita englobar mais alunos a custos menores
	e com boa qualidade \cite{schmidt2013producing}.
	
	Em cursos de introdução a programação, seja \acs{MOOC} ou presencial, há uma introdução
	sobre algoritmo seguido da seleção de uma linguagem de programação para
	desenvolvimento. Alguns desses cursos orientam a utilização de ferramentas, como
	o \ac{IDE} e ensinam o funcionamento da linguagem sintaticamente.
	Entretanto, esse tipo de ensino tem levado o estudante a fazer seu programa baseado
	na tentativa e erro, visto que buscam gerar o algoritmo sem o total conhecimento
	lógico do problema \cite{edwards2003}. Isso funcionou como um incentivo para que
	surgissem outras abordagens de ensino, como a reflexão na ação (\foreign{Reflection
	in action}) e o \ac{TDD} \cite{camara_graciottoSilva2016}. Na primeira abordagem, a
	implementação do \foreign{software} ocorre somente após o entendimento apropriado
	do problema \cite{edwards2004}. No \acs{TDD}, cumpre-se um ciclo dividido em três etapas:
	na primeira etapa o programador insere um teste que deve falhar no momento de sua
	execução; na segunda etapa é implementado a solução para que o caso de teste
	escrito anteriormente seja aceito; e a última etapa refere-se a refatoração
	(simplificar ou melhorar) o código recém implementado \cite{beck2003}.