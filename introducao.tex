\chapter{Introdução}

	MOOC - \foreign{Massive Open Online Course} - é uma plataforma de ensino online
	com cursos massivos e abertos que visa oferecer os mais diversos cursos a nível
	global. Para realizar um curso ofertado pelo MOOC, é necessário apenas a conexão
	com a Internet e o cadastro na plataforma, geralmente gratuito. Foi popularizado
	em 2011 quando grandes universidades, como o Instituto de Tecnologia de
	Massachusetts, a Universidade de Harvard e a Universidade de Stanford tiveram
	a iniciativa mediados pelos provedores Cousera, edX e Udacity, respectivamente
	\cite{Mehlenbacher:2012}. Além de permitir explorar novos modelos de negócio
	\cite{dellarocas2013money}, possibilita englobar mais alunos a custos menores
	e com boa qualidade \cite{schmidt2013producing}.
	
	Em cursos de introdução a programação, seja MOOC ou presencial, há uma introdução
	sobre algoritmo seguido da seleção de uma linguagem de programação para
	desenvolvimento. Alguns desses cursos orientam a utilização de ferramentas, como
	o ambiente integrado de desenvolvimento (IDE, do inglês \foreign{Integrated
	Development Environment}) e ensinam o funcionamento da linguagem sintaticamente.
	Entretanto, esse tipo de ensino tem levado o estudante a fazer seu programa baseado
	na tentativa e erro, visto que buscam gerar o algoritmo sem o total conhecimento
	lógico do problema \cite{edwards2003}. Isso funcionou como um incentivo para que
	surgissem outras abordagens de ensino, como a reflexão na ação (\foreign{Reflection
	in action}) e o desenvolvimento baseado em teste (TDD, do inglês \foreign{Test
	Driven Development}) \cite{camara_graciottoSilva2016}. Na primeira abordagem, a
	implementação do \foreign{software} ocorre somente após o entendimento apropriado
	do problema \cite{edwards2004}. No TDD, cumpre-se um ciclo dividido em três etapas:
	na primeira etapa o programador insere um teste que deve falhar no momento de sua
	execução; na segunda etapa é implementado a solução para que o caso de teste
	escrito anteriormente seja aceito; e a última etapa refere-se a refatoração
	(simplificar ou melhorar) o código recém implementado \cite{beck2003}.
	
	% Problema de pesquisa
	Principalmente em MOOC de ensino de programação, tratando-se de uma ferramenta de
	ensino a nível mundial, deve-se considerar um grande número de usuários. Com isso,
	torna-se necessário a utilização de mecanismos de avaliação automática ou
	semiautomática \cite{schmidt2013producing}. Para alguns tipos de atividade, a
	avaliação automática é bem simples. Por exemplo, para verificar as soluções
	referentes às questões de múltipla escolha é necessário somente comparar a
	alternativa selecionada com o gabarito de questões \cite{alario2013analysing}. No
	entanto, implementações de programas computacionais necessitam que seus algoritmos
	sejam analisados quanto às saídas geradas pela sua execução, projeto do algoritmo,
	facilidade de compreensão, dentro outros requisitos, a fim de retornar o resultado
	da avaliação. Como há uma grande quantidade de submissões de trabalhos em cursos
	de programação, consequentemente acarreta alguns problemas como: avaliar todas as
	submissões, tempo e aumento de custo de correção por parte do professor.
	
	% Hipóteses
	
	% Justificativa
	
	% Objetivo
	Considerando que muito projetos podem ser semelhantes, os professores podem explorar
	e compreender as variações de implementações \cite{Yin:2015}, a fim de diminuir
	o tempo gasto para correção dos códigos-fontes submetidos, por meio de agrupamentos
	(\foreign{clusters}). Por exemplo, tais agrupamentos podem ser realizados por meio
	da similaridade dos códigos e agrupados com um algoritmo como o \foreign{k-means}.
	Compreendendo as variações de implementação, é possível extrair caraterísticas por
	meio da análise estática \cite{Yin:2015,Glassman:2014,Taherkhani:2012}, análise
	dinâmica \cite{Glassman:2015} e análise do estilo de escrita \cite{Wei2015}.
	Com	isso, o agrupamento das submissões semelhantes infere na correção de poucos
	projetos, já que todos os outros códigos que estão no agrupamento serão
	parecidos, podendo melhorar a qualidade de vida do professor, a eficiência da correção
	por meio de \foreign{feedbacks} mais precisos e menor custo para o MOOC.

%	\citeonline{Yin:2015} extraem uma árvore de sintaxe abstrata de cada implementação
%	e verifica a similaridade par a par por meio da distância de edição de árvore. Tal
%	abordagem pode agrupar algoritmos corretos e incorretos, por exemplo: pode ocorrer
%	erro lógico e uma ou mais instruções estarem invertidas, o cálculo de similaridade
%	pode apontar que as implementações são semelhantes e agrupar códigos-fontes corretos
%	com incorretos. Entretanto, essa abordagem possui baixo custo computacional e
%	interessante para possibilidade de adaptação para verificar erros lógicos.
%	
%	\citeonline{Glassman:2014} extraem 60 características de submissões implementadas
%	em Python separadas em duas dimensões: alto nível e baixo nível. Realiza o agrupamento
%	utilizando as características de alto nível e, internamente desses agrupamentos,
%	executa o \foreign{k-means} novamente para as características de baixo nível. Apesar
%	de separar características de alto e baixo nível. Essa abordagem não considera as
%	chamadas de função e recursões, com isso o algoritmo, caso utilize essas instruções,
%	pode não ser classificado. Entretanto, o agrupamento em dois níveis é interessante
%	devido a possibilidade de obter agrupamentos com erros semelhantes apenas das
%	características de nível alto e, após a realização do segundo agrupamento, das
%	características de nível baixo, permitindo a realização do \foreign{feedback} em
%	duas etapas.
%	
%	\citeonline{Taherkhani:2012} apresentam a ferramenta Aari para classificar e reconhecer
%	algoritmos de ordenação. Para isso, extrai diversas características: numéricas,
%	descritivas, outras e de algoritmos de ordenação. Entretanto, é necessário que
%	treine a ferramenta para reconhecer os algoritmos desejados, caso contrário, o
%	classificador não estará apto e ocorrerá diversos erros de classificação. Sua
%	abordagem é interessante pelo fato de extrair características referente a recursão.
%	
%	\citeonline{Glassman:2015} apresentam a ferramenta OverCode que extrai características
%	por meio do histórico de execução da análise dinâmica. A similaridade das submissões
%	é realizada por meio da comparação entre blocos do programa. A partir disso é utilizado
%	o conceito de pilha para apresentar o agrupamentos das implementações semelhantes e,
%	ao clicar em um agrupamento, é possível verificar as linhas de código que formara
%	aquele agrupamento. Essa abordagem proporcionou o desenvolvimento de vários recursos
%	na ferramenta, como reescrever as variáveis que possuem a mesma atribuição, e mostrou
%	que a combinação de blocos de código e análise dinâmica pode produzir bons agrupamentos.
%	
%	\citeonline{Wei2015} realizam a extração de características por meio da normalização
%	do código-fonte e da análise do estilo de escrita. Para isso, separou cada implementação
%	em pedaços de código, condizentes a uma função, para verificar a similaridade entre eles.
%	Com isso, o instrutor corrigiria pedaços do código ao invés de corrigir todo o algoritmo.
%	O autor encontra a menor substring do bloco a partir de comparações \foreign{token} a
%	\foreign{token} para verificar a similaridade. Limitou-se a utilizar os pedaços de
%	código, sendo que poderia ter comparado com os agrupamentos considerando toda a
%	implementação. Entretanto, sua normalização foi interessante e a pouca quantidade de
%	características extraídas pode ter sido primordial para formar os agrupamentos.
	
	Considerando este cenário, o objetivo deste trabalho é propor subsídios para a
	avaliação de programas submetidos em MOOC, utilizando técnicas de mineração e visualização
	de dados com a finalidade de diminuir o tempo gasto na correção de todos os códigos
	fontes e  fornecer \foreign{feedbacks} construtivos de modo que o usuário consiga
	corrigir seus erros e submeta novamente seu algoritmo. Como metas, estabeleceu-se
	o desenvolvimento de uma ferramenta para recuperação de dados a partir de
	códigos-fontes e da alteração de uma ferramenta para mineração e visualização de
	dados~\cite{Alencar}. Com o auxílio dessas ferramentas, serão investigadas as
	características e técnicas para mineração e visualização dos programas submetidos,
	avaliando-se como contribuir para a correção dos trabalhos submetidos, o tempo em
	que foi necessário para que os professores corrigissem todas as submissões, a
	qualidade dos agrupamentos realizados pelo sistema e, principalmente, a
	qualidade do \foreign{feedback}.
	
	O restante desta proposta de trabalho de conclusão de curso é organizado da seguinte
	forma. O \cref{chap:Ref} apresenta o referencial teórico, divido em 4 seções: a
	\cref{sec:Mooc} apresenta algumas definições de MOOC e suas características; a
	\cref{sec:FundTeor} trata dos principais conceitos referentes à mineração e
	visualização de dados utilizados neste trabalho; a \cref{sec:MinVisual} cita as
	formas que podem ser realizados a extração de características e quais os possíveis
	dados a serem coletados conforme o tipo de análise utilizado; e a \cref{sec:TrabRel}
	descreve os trabalhos relacionados que fundamentaram o projeto. O \cref{chap:Proposta}
	apresenta a método de pesquisa para obter os resultados, seguido do \cref{chap:Result}
	que possui os resultados preliminares.